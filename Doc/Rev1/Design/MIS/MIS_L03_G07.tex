\documentclass[12pt]{article}

\usepackage{tabularx}
\usepackage{booktabs}
\usepackage{graphicx}
\usepackage{paralist}
\usepackage{listings}
\usepackage{booktabs}
\usepackage{hyperref}
\usepackage{amsfonts}
\usepackage{amsmath}
\usepackage{color}
\usepackage{fancyhdr}
\usepackage{geometry}
\usepackage{multirow}
\usepackage[round]{natbib}
\usepackage{soul}

\geometry{margin = 0.75in}
\title{SE 3XA3: MIS\\Space Invaders}

\begin{document}

\maketitle

%%%%%%%%%%%%%%%%%%%Team Information%%%%%%%%%%%%%%%%%%%%%%
{\Large Team Information:}
\begin{table}[htp]
\centering
{\Large
\begin{tabular}{|c|c|c|}
\hline
\multicolumn{1}{|l|}{Team Number} & Name         & MACID   \\ \hline
\multirow{3}{*}{L03 G07}          & Qianlin Chen & chenq84 \\ \cline{2-3} 
                                  & Jiacheng Wu  & wuj187  \\ \cline{2-3} 
                                  & Tingyu Shi   & shit19  \\ \hline
\end{tabular}
}
\end{table}

%%%%%%%%%%% Revision History%%%%%%%%%%%%%
\newpage
\begin{table}[htp]
\caption{Revision History} 
\begin{tabularx}{\textwidth}{llX}
\toprule
\textbf{Date} & \textbf{Developer(s)} & \textbf{Change}\\
\midrule
January 26, 2022 & All team members & Initial Document\\
March 18, 2022 & Qianlin Chen & Display Modules\\
March 10, 2022 & Jiacheng Wu & Control Modules\\
March 10, 2022 & Tingyu Shi & Model modules\\
April 11, 2022 & All team members & \textcolor{red}{Revised document}\\
\bottomrule
\end{tabularx}
\end{table}
\newpage
%%%%%%%%%%%%%%Contents%%%%%%%%%%%%%%%%
\tableofcontents
\listoftables
\listoffigures
\cleardoublepage

%%%%%%%%%%%%%%%%%%%%%%%%%%%%%%%%%%%MonsterColor%%%%%%%%%%%%%%%%
\section{MonsterColor Module}

\subsection*{Module}
MonsterColor

\subsection*{Uses}
None

\subsection*{Syntax}
\subsubsection*{Exported Constants}
None
\subsubsection*{Exported Types}
MonsterType = $\{RED, BLUE, PINK\}$ \# Represent Three colors of monsters
\subsubsection*{Exported Access Programs}
None (This is an Enum class in python)

\subsection*{Semantics}
\subsubsection*{State Variables}
None
\subsubsection*{State Invariant}
None
\subsubsection*{Assumptions}
None
\subsubsection*{Access Routine Semantics}
None
\subsubsection*{Consideration}
When implementing in Python, use Enum class.
\newpage
%%%%%%%%%%%%%%%%%%%%%%%%%%%%%%%%%%%%%%%%%%%%%%%%%%%%%%%%%%%%%%%%%

%%%%%%%%%%%%%%%%%%%%%%%%%BulletState%%%%%%%%%%%%%%%%%%%%%%%%%%%%%
\section{BulletState Module}

\subsection*{Module}
BulletState

\subsection*{Uses}
None

\subsection*{Syntax}
\subsubsection*{Exported Constants}
None
\subsubsection*{Exported Types}
BulletState = $\{FIRE, READY\}$ \# Represent two states of bullets
\subsubsection*{Exported Access Programs}
None (This is an Enum class in python)

\subsection*{Semantics}
\subsubsection*{State Variables}
None
\subsubsection*{State Invariant}
None
\subsubsection*{Assumptions}
None
\subsubsection*{Access Routine Semantics}
None
\subsubsection*{Consideration}
When implementing in Python, use Enum class.
\newpage
%%%%%%%%%%%%%%%%%%%%%%%%%%%%%%%%%%%%%%%%%%%%%%%%%%%%%%%%%%%%%%%%%


%%%%%%%%%%%%%%%%%%%%%%%Monster%%%%%%%%%%%%%%%%%%%%%%%%%%%%%%%%%%%
\section{Monster Module}

\subsection*{Template Module}
Monster

\subsection*{Uses}
MonsterColor, Bullet, \textcolor{red}{pygame.sprite.Sprite}

\subsection*{Syntax}
\subsubsection*{Exported Constants}
None
\subsubsection*{Exported Types}
Monster = ?
\subsubsection*{Exported Access Programs}
\begin{tabular}{| l | l | l | p{5cm} |}
\hline
\textbf{Routine name} & \textbf{In} & \textbf{Out} & \textbf{Exceptions}\\
\hline
new Monster & $\mathbb{R}$, $\mathbb{R}$, $\mathbb{R}$, $MonsterColor$,  $\mathbb{R}$& Monster & IllegalArgumentException\\
\hline
\textcolor{red}{\st{
setX} Method removed} & $\mathbb{R}$ &  & IllegalArgumentException\\
\hline
\textcolor{red}{\st{
setY } Method removed}& $\mathbb{R}$ &  & IllegalArgumentException\\
\hline
\textcolor{red}{\st{
getX} Method removed} &    & $\mathbb{R}$ & \\
\hline
\textcolor{red}{\st{
getY } Method removed}&    & $\mathbb{R}$ & \\
\hline
\textcolor{red}{\st{
getColor }Method removed}& & $MonsterColor$ &\\
\hline
reduceLife & & & \\
\hline
isDead & & $\mathbb{B}$ & \\
\hline
update & $\mathbb{Z}$ & &\\
\hline
\textcolor{red}{\st{
shoot} Method removed} & & $Bullet$ &\\
\hline
getItemType & & $\mathbb{Z}$ &\\
\hline
\end{tabular}

\subsection*{Semantics}
\subsubsection*{State Variables}
$\mathit{speed}$: $\mathbb{R}$\\
\textcolor{red}{\st{$\mathit{X}$: $\mathbb{R}$}}\\
\textcolor{red}{\st{$\mathit{Y}$: $\mathbb{R}$}}\\
\textcolor{red}{\st{$\mathit{monster\_color}$: $MonsterColor$}}\\
\textcolor{red}{\st{$\mathit{X\_change}$: $\mathbb{R}$}}\\
\textcolor{red}{\st{$\mathit{Y\_change}$: $\mathbb{R}$}}\\
$\mathit{life}$: $\mathbb{Z}$\\
$\mathit{itemType}$: $\mathbb{Z}$\\
%%%%%%%%%%%%%reuse
$\mathit{image}$: \textit{.png file}\\
$\mathit{rect}$: \textit{image.get\_rect (This is the API of pygame library)}
%%%%%%%%%%%%%%%%%%

\subsubsection*{State Invariant}
\textcolor{red}{\st{$0 \leq X \leq 736$}}
\subsubsection*{Assumptions}
None
\subsubsection*{Access Routine Semantics}
\noindent new Monster($\mathit{x}, \mathit{y}, \mathit{color}, \mathit{s}, $):
\begin{itemize}
\item transition: \\$\mathit{speed}, \mathit{itemType} := \mathit{s}, 1$\\
  $color = MonsterColor.GREEN \Rightarrow  life := 1$\\
  $color = MonsterColor.BLUE \Rightarrow  life := 2$\\
  $color = MonsterColor.PINK \Rightarrow  life := 3$\\
  $image := \textit{corresponding image}$\\
  $rect := image.get\_rect(topleft\ =\ (x,\ y))$
\item output: $out := \mathit{self}$
\item exception: exc := $((\mathit{s} < 0) \vee (\mathit{x} < 0) \vee 
  (\mathit{y} < 0) \Rightarrow \text{IllegalArgumentException})$
\end{itemize}

\noindent \textcolor{red}{\st{setX($\mathit{x}$)} Method removed}:
\begin{itemize}
\item transition: $X := x$ 
\item output: none
\item exception: $((\mathit{x} < 0) \Rightarrow \text{IllegalArgumentException})$
\end{itemize}

\noindent \textcolor{red}{\st{setY($\mathit{y}$)} Method removed}:
\begin{itemize}
\item transition: $Y := y$ 
\item output: none
\item exception: $((\mathit{y} < 0) \Rightarrow \text{IllegalArgumentException})$
\end{itemize}

\noindent \textcolor{red}{\st{getX()} Method removed}:
\begin{itemize}
\item transition: none
\item output: $out := X$
\item exception: none
\end{itemize}

\noindent \textcolor{red}{\st{getY()} Method removed}:
\begin{itemize}
\item transition: none
\item output: $out := Y$
\item exception: none
\end{itemize}

\noindent \textcolor{red}{\st{getColor()} Method removed}:
\begin{itemize}
\item transition: none
\item output: $out := monster\_color$
\item exception: none
\end{itemize}

\noindent reduceLife():
\begin{itemize}
\item transition: $life := life - 1$
\item output: none
\item exception: none
\end{itemize}

\noindent isDead():
\begin{itemize}
\item transition: none
\item output: $life = 0$
\item exception: none
\end{itemize}

\noindent update($direction$):
\begin{itemize}
\item transition: \\
$rect.x := rect.x\ +\ (direction \times speed)$
\item output: none
\item exception: none
\end{itemize}

\noindent \textcolor{red}{\st{shoot()} Method removed}:
\begin{itemize}
\item transition: none
\item output: $new\ Bullet(20,\ X,\ Y)$
\item exception: none
\end{itemize}

\noindent getItemType():
\begin{itemize}
\item transition: none
\item output: $itemType$
\item exception: none
\end{itemize}
\newpage
%%%%%%%%%%%%%%%%%%%%%%%%%%%%%%%%%%%%%%%%%%%%%%%%%%%%%%%%%%%%%%%%%

%%%%%%%%%%%%%%%%%%%%%%%%%%%%%%%SpaceShip%%%%%%%%%%%%%%%%%%%%%%%%
\section{SpaceShip Module}

\subsection*{Template Module}
SpaceShip

\subsection*{Uses}
Bullet, pygame.sprite.Sprite
\subsection*{Syntax}
\subsubsection*{Exported Constants}
None
\subsubsection*{Exported Types}
SpaceShip = ?
\subsubsection*{Exported Access Programs}
\begin{tabular}{| l | l | l | p{5cm} |}
\hline
\textbf{Routine name} & \textbf{In} & \textbf{Out} & \textbf{Exceptions}\\
\hline
new Monster & $\mathbb{R}$, $\mathbb{R}$, ($\mathbb{R}, \mathbb{R}$), $\mathbb{R}$, $\mathbb{Z}$ & SpaceShip & IllegalArgumentException\\
\hline
\textcolor{red}{\st{setX} Method removed} & $\mathbb{R}$ &  & IllegalArgumentException\\
\hline
\textcolor{red}{\st{setY} Method removed} & $\mathbb{R}$ &  & IllegalArgumentException\\
\hline
\textcolor{red}{\st{getX} Method removed} &    & $\mathbb{R}$ & \\
\hline
\textcolor{red}{\st{getY} Method removed} &    & $\mathbb{R}$ & \\
\hline
\textcolor{red}{\st{moveLeft} Method removed} & & &\\
\hline
\textcolor{red}{\st{moveRight} Method removed} & & &\\
\hline
\textcolor{red}{\st{stopMove} Method removed} & & &\\
\hline
reduceLife & & & \\
\hline
\textcolor{red}{\st{isDead} Method removed} & & $\mathbb{B}$ & \\
\hline
boundaryDetection & & &\\
\hline
shoot & & $Bullet$ &\\
\hline
update &&&\\
\hline
move &&&\\
\hline
prepare\_bullet &&&\\
\hline
getBulletsGroup && pygame sprite group &\\
\hline
increaseLife &&&\\
\hline
getLife &&$\mathbb{Z}$&\\
\hline
setLife &$\mathbb{Z}$&&\\
\hline
increaseBullet &&&\\
\hline
\end{tabular}

\subsection*{Semantics}
\subsubsection*{State Variables}
$\mathit{screen\_size\_info}$: $(\mathbb{R}, \mathbb{R})$\\
$\mathit{space\_ship\_number}$: $\mathbb{Z}$\\
$\mathit{speed}$: $\mathbb{R}$\\
\textcolor{red}{\st{$\mathit{X}$: $\mathbb{R}$}}\\
\textcolor{red}{\st{$\mathit{Y}$: $\mathbb{R}$}}\\
\textcolor{red}{\st{$\mathit{X\_change}$: $\mathbb{R}$}}\\
$\mathit{life}$: $\mathbb{Z}$\\
$\mathit{image}$: \textit{.png file}\\
$\mathit{rect}$: \textit{image.get\_rect (This is the API of pygame library)}\\
bullets\_group: pygame sprite group\\
state: BulletState\\
shoot\_time: $\mathbb{Z}$\\
bullet\_number: $\mathbb{Z}$
\subsubsection*{State Invariant}
\textcolor{red}{\st{$0 \leq X \leq 736$}}\\
$0 \leq life \leq 5$
\subsubsection*{Assumptions}
None
\subsubsection*{Access Routine Semantics}
\noindent new SpaceShip(x, y, size, s, number):
\begin{itemize}
\item transition:
screen\_size\_info := size\\
space\_ship\_number := number\\
$image := \textit{corresponding image}$\\
  $rect := image.get\_rect(midbottom\ =\ (x,\ y))$\\
  speed := s\\
  life := 5\\
  bullets\_group := new Pygame sprite group\\
  state := BulletState.READY\\
  shoot\_time := 0\\
  bullet\_number := 0
\item output: $out := \mathit{self}$
\item exception: exc := $((\mathit{s} < 0) \vee (\mathit{x} < 0) \vee 
  (\mathit{y} < 0) \Rightarrow \text{IllegalArgumentException})$
\end{itemize}

\noindent \textcolor{red}{\st{setX($\mathit{x}$)}}:
\begin{itemize}
\item transition: $X := x$ 
\item output: none
\item exception: $((\mathit{x} < 0) \Rightarrow \text{IllegalArgumentException})$
\end{itemize}

\noindent \textcolor{red}{\st{setY($\mathit{y}$)}}:
\begin{itemize}
\item transition: $Y := y$ 
\item output: none
\item exception: $((\mathit{y} < 0) \Rightarrow \text{IllegalArgumentException})$
\end{itemize}

\noindent \textcolor{red}{\st{getX()}}:
\begin{itemize}
\item transition: none
\item output: $out := X$
\item exception: none
\end{itemize}

\noindent \textcolor{red}{\st{getY()}}:
\begin{itemize}
\item transition: none
\item output: $out := Y$
\item exception: none
\end{itemize}

\noindent \textcolor{red}{\st{moveLeft()}}:
\begin{itemize}
\item transition: $X\_change := -1 * speed$
\item output: none
\item exception: none
\end{itemize}

\noindent \textcolor{red}{\st{moveRight()}}:
\begin{itemize}
\item transition: $X\_change := speed$
\item output: none
\item exception: none
\end{itemize}

\noindent \textcolor{red}{\st{stopMove()}}:
\begin{itemize}
\item transition: $X\_change := 0$
\item output: none
\item exception: none
\end{itemize}

\noindent reduceLife():
\begin{itemize}
\item transition: $life := life - 1$
\item output: none
\item exception: none
\end{itemize}

\noindent \textcolor{red}{\st{isDead()}}:
\begin{itemize}
\item transition: none
\item output: $life = 0$
\item exception: none
\end{itemize}

\noindent boundaryDetection():
\begin{itemize}
\item transition: \\
$rect.left \leq 0 \Rightarrow (rect.left := 0)$\\
$rect.right \geq screen\_size\_info[0] \Rightarrow (rect.right := screen\_size\_info[0])$
\item output: none
\item exception: none
\end{itemize}

\noindent shoot():
\begin{itemize}
\item transition: none
\item output: add bullet(s) to bullet\_group according to user keyboard input
\item exception: none
\end{itemize}

\noindent update():
\begin{itemize}
\item transition: \\move()\\boundaryDetection()\\shoot()\\prepare\_bullet()\\bullets\_group.update()
\item output: none
\item exception: none
\end{itemize}


\noindent \textcolor{red}{move()}:
\begin{itemize}
\item transition: \\
rect.x += speed if user wants to move right\\
rect.x -= speed if user wants to move left
\item output: none
\item exception: none
\end{itemize}

\noindent \textcolor{red}{prepare\_bullet()}:
\begin{itemize}
\item transition: set up the gap time  between two shootings
\item output: none
\item exception: none
\end{itemize}

\noindent \textcolor{red}{getBulletsGroup()}:
\begin{itemize}
\item transition: none
\item output: out := bullets\_group
\item exception: none
\end{itemize}

\noindent \textcolor{red}{increaseLife()}:
\begin{itemize}
\item transition: life := life + 1
\item output: none
\item exception: none
\end{itemize}

\noindent \textcolor{red}{getLife()}:
\begin{itemize}
\item transition: none
\item output: out := life
\item exception: none
\end{itemize}

\noindent \textcolor{red}{setLife(newLife)}:
\begin{itemize}
\item transition: life := newLife
\item output: none
\item exception: none
\end{itemize}

\noindent \textcolor{red}{increaseBullet()}:
\begin{itemize}
\item transition: none
\item output: bullet\_number := bullet\_number + 1
\item exception: none
\end{itemize}

\newpage
%%%%%%%%%%%%%%%%%%%%%%%%%%%%%%%%%%%%%%%%%%%%%%%%%%%%%%%%%%%%%%%%

%%%%%%%%%%%%%%%%%%%%%%%%%%%%%%%%%%Bullet%%%%%%%%%%%%%%%%%%%%%%%%
\section{Bullet Module}

\subsection*{Template Module}
Bullet

\subsection*{Uses}
pygame.sprite.Sprite

\subsection*{Syntax}
\subsubsection*{Exported Constants}
None
\subsubsection*{Exported Types}
Bullet = ?
\subsubsection*{Exported Access Programs}
\begin{tabular}{| l | l | l | p{5cm} |}
\hline
\textbf{Routine name} & \textbf{In} & \textbf{Out} & \textbf{Exceptions}\\
\hline
new Bullet & $\mathbb{R}$, $\mathbb{R}$, ($\mathbb{R}$, $\mathbb{R}$), $\mathbb{R}$, $\mathbb{Z}$  & Bullet & IllegalArgumentException\\
\hline
\textcolor{red}{\st{
setX}Method removed} & $\mathbb{R}$ &  & IllegalArgumentException\\
\hline
\textcolor{red}{\st{
setY}Method removed} & $\mathbb{R}$ &  & IllegalArgumentException\\
\hline
\textcolor{red}{\st{
getX}Method removed} &    & $\mathbb{R}$ & \\
\hline
\textcolor{red}{\st{
getY}Method removed} &    & $\mathbb{R}$ & \\
\hline
\textcolor{red}{\st{
setState}Method removed} & $BulletState$ & &\\
\hline
\textcolor{red}{\st{
getState}Method removed} & & $BulletState$ &\\
\hline
move & & &\\
\hline
boundaryDetection &&&\\
\hline
update &&&\\
\hline

\end{tabular}

\subsection*{Semantics}
\subsubsection*{State Variables}
screen\_size\_info : ($\mathbb{R}, \mathbb{R}$)\\
$\mathit{speed}$: $\mathbb{R}$\\
\textcolor{red}{\st{
$\mathit{X}$: $\mathbb{R}$}}\\
\textcolor{red}{\st{
$\mathit{Y}$: $\mathbb{R}$}}\\
\textcolor{red}{\st{
$\mathit{Y\_change}$: $\mathbb{R}$}}\\
\textcolor{red}{\st{
$\mathit{state}$: $BulletState$}}\\
$\mathit{image}$: \textit{.png file}\\
$\mathit{rect}$: \textit{image.get\_rect (This is the API of pygame library)}
\subsubsection*{State Invariant}
none
\subsubsection*{Assumptions}
None
\subsubsection*{Access Routine Semantics}
\noindent new Bullet($\mathit{x}, \mathit{y}, \mathit{size}$, s, direction):
\begin{itemize}
\item transition:\\ screen\_size\_info := size \\ speed := s\\$image := \textit{corresponding image(direction is the condition)}$\\
 $rect := image.get\_rect(center\ =\ (x,\ y))$

\item output: $out := \mathit{self}$
\item exception: exc := $((\mathit{s} < 0) \vee (\mathit{x} < 0) \vee 
  (\mathit{y} < 0) \Rightarrow \text{IllegalArgumentException})$
\end{itemize}

\noindent \textcolor{red}{\st{
setX($\mathit{x}$)}}:
\begin{itemize}
\item transition: $X := x$ 
\item output: none
\item exception: $((\mathit{x} < 0) \Rightarrow \text{IllegalArgumentException})$
\end{itemize}

\noindent \textcolor{red}{\st{
setY($\mathit{y}$)}}:
\begin{itemize}
\item transition: $Y := y$ 
\item output: none
\item exception: $((\mathit{y} < 0) \Rightarrow \text{IllegalArgumentException})$
\end{itemize}

\noindent  \textcolor{red}{\st{
getX()}}:
\begin{itemize}
\item transition: none
\item output: $out := X$
\item exception: none
\end{itemize}

\noindent \textcolor{red}{\st{
getY()}}:
\begin{itemize}
\item transition: none
\item output: $out := Y$
\item exception: none
\end{itemize}

\noindent \textcolor{red}{\st{
setState($\mathit{newState}$)}}:
\begin{itemize}
\item transition: $state := newState$ 
\item output: none
\item exception: none
\end{itemize}

\noindent \textcolor{red}{\st{
getState()}}:
\begin{itemize}
\item transition: none
\item output: $out := state$
\item exception: none
\end{itemize}

\noindent move():
\begin{itemize}
\item transition: rect.y := rect.y + speed
\item output: none
\item exception: none
\end{itemize}


\noindent boundaryDetection():
\begin{itemize}
\item transition: rect.y $\leq$ 30 OR rect.y $\geq$ screen\_size\_info[1] $\Rightarrow$
kill self object
\item output: none
\item exception: none
\end{itemize}

\noindent update():
\begin{itemize}
\item transition: \\move()\\boundaryDetection()
\item output: none
\item exception: none
\end{itemize}

\newpage
%%%%%%%%%%%%%%%%%%%%%%%%%%%%%%%%%%%%%%%%%%%%%%%%%%%%%%%%%%%%%%%%

%%%%%%%%%%%%%%%%%%%%%%%%%%%%%%%Score%%%%%%%%%%%%%%%%%%%%%%%%%%%%
\section{Score Module}

\subsection*{Template Module}
Score

\subsection*{Uses}
none

\subsection*{Syntax}
\subsubsection*{Exported Constants}
None
\subsubsection*{Exported Types}
Score = ?
\subsubsection*{Exported Access Programs}
\begin{tabular}{| l | l | l | p{5cm} |}
\hline
\textbf{Routine name} & \textbf{In} & \textbf{Out} & \textbf{Exceptions}\\
\hline
new score &  & Score &\\
\hline
getScore &    & $\mathbb{N}$ & \\
\hline
\textcolor{red}{\st{
increaseAmount} increase} & $\mathbb{N}$ & &\\
\hline
\end{tabular}

\subsection*{Semantics}
\subsubsection*{State Variables}
$\mathit{score}$: $\mathbb{N}$
\subsubsection*{State Invariant}
$score \geq 0$
\subsubsection*{Assumptions}
None
\subsubsection*{Access Routine Semantics}
\noindent new Score():
\begin{itemize}
\item transition: $\mathit{score} := 0$
\item output: $out := \mathit{self}$
\item exception: exc := none
\end{itemize}
\newpage

\noindent getScore():
\begin{itemize}
\item transition: none
\item output: $out := score$
\item exception: none
\end{itemize}

\noindent \textcolor{red}{\st{
increaseAmount} increase}($\mathit{amount}$):
\begin{itemize}
\item transition: $score := score + amount$ 
\item output: none
\item exception: none
\end{itemize}
\newpage
%%%%%%%%%%%%%%%%%%%%%%%%%%%%%%%%%%%%%%%%%%%%%%%%%%%%%%%%%%%%%%%%

%%%%%%%%%%%%%%%%%%%%%%%%%%%%Block%%%%%%%%%%%%%%%%%%%%%%%%%%%%
\section{Block Module\textcolor{red}{(newly added module)}}

\subsection*{Template Module}
Block

\subsection*{Uses}
pygame.sprite.Sprite

\subsection*{Syntax}
\subsubsection*{Exported Constants}
None
\subsubsection*{Exported Types}
Obstacle = ?
\subsubsection*{Exported Access Programs}
\begin{tabular}{| l | l | l | p{5cm} |}
\hline
\textbf{Routine name} & \textbf{In} & \textbf{Out} & \textbf{Exceptions}\\
\hline
new Block & $\mathbb{R}$, $\mathbb{R}$ & Obstacle &\\
\hline
\end{tabular}

\subsection*{Semantics}
\subsubsection*{State Variables}
$\mathit{image}$: \textit{.png file}\\
$\mathit{rect}$: \textit{image.get\_rect (This is the API of pygame library)}
\subsubsection*{State Invariant}
None
\subsubsection*{Assumptions}
None
\subsubsection*{Access Routine Semantics}
\noindent new Block($\mathit{x}, \mathit{y}$):
\begin{itemize}
\item transition:\\
$image := \textit{corresponding image}$\\
 $rect := image.get\_rect(topleft\ =\ (x,\ y))$
\item output: $out := \mathit{self}$
\item exception: exc := none
\end{itemize}
\newpage
%%%%%%%%%%%%%%%%%%%%%%%%%%%%%%%%%%%%%%%%%%%%%%%%%%%%%%%%%%%%%%%%

%%%%%%%%%%%%%%%%%%%%%%%%%%%%Obstacle%%%%%%%%%%%%%%%%%%%%%%%%%%%%
\section{Obstacle Module}

\subsection*{Template Module}
Obstacle

\subsection*{Uses}
Block, pygame.sprite.Sprite

\subsection*{Syntax}
\subsubsection*{Exported Constants}
None
\subsubsection*{Exported Types}
Obstacles = ?
\subsubsection*{Exported Access Programs}
\begin{tabular}{| l | l | l | p{5cm} |}
\hline
\textbf{Routine name} & \textbf{In} & \textbf{Out} & \textbf{Exceptions}\\
\hline
new Obstacle & & Obstacle &\\
\hline
creat\_one\_obstacle & $\mathbb{Z}, \mathbb{Z}$   &  & \\
\hline
getBlocksGroup &    & pygame sprite group & \\
\hline
\end{tabular}

\subsection*{Semantics}
\subsubsection*{State Variables}
\textcolor{red}{\st{$\mathit{X}$: $\mathbb{R}$}}\\
\textcolor{red}{\st{$\mathit{Y}$: $\mathbb{R}$}}\\
\textcolor{red}{\st{$\mathit{Width}$: $\mathbb{R}$}}\\
\textcolor{red}{\st{$\mathit{Height}$: $\mathbb{R}$}}\\
\textcolor{red}{\st{$\mathit{Area}$: $\mathbb{R}$}}\\
blocks\_group := pygame sprite group
\subsubsection*{State Invariant}
None
\subsubsection*{Assumptions}
None
\subsubsection*{Access Routine Semantics}
\noindent new Obstacle():
\begin{itemize}
\item transition: \\
create\_one\_obstacle(50, 400)\\
create\_one\_obstacle(250, 400)\\
create\_one\_obstacle(450, 400)\\
create\_one\_obstacle(650, 400)
\item output: $out := \mathit{self}$
\item exception: exc := none
\end{itemize}

\noindent create\_one\_obstacle(xStart, yStart):
\begin{itemize}
\item transition: create blocks starting at (xStart, yStart) and add all
the blocks to blocks\_group
\item output: $out := \mathit{self}$
\item exception: exc := none
\end{itemize}

\noindent getBlocksGroup():
\begin{itemize}
\item transition: none
\item output: $out := \mathit{blocks\_group}$
\item exception: exc := none
\end{itemize}


\noindent \textcolor{red}{\st{getX()}}:
\begin{itemize}
\item transition: none
\item output: $out := X$
\item exception: none
\end{itemize}

\noindent \textcolor{red}{\st{getY()}}:
\begin{itemize}
\item transition: none
\item output: $out := Y$
\item exception: none
\end{itemize}

\noindent \textcolor{red}{\st{getArea()}}:
\begin{itemize}
\item transition: none
\item output: $out := Area$
\item exception: none
\end{itemize}

\noindent \textcolor{red}{\st{reduce\_area($\mathit{amount}$)}}:
\begin{itemize}
\item transition: $Area := Area - amount$ 
\item output: none
\item exception: none
\end{itemize}
\newpage
%%%%%%%%%%%%%%%%%%%%%%%%%%%%%%%%%%%%%%%%%%%%%%%%%%%%%%%%%%%%%%%%

%%%%%%%%%%%%%%%%%%%%%%%%%%%%%%%Ammo%%%%%%%%%%%%%%%%%%%%%%%%%%%%%
\section{Ammo Module}

\subsection*{Template Module}
Ammo

\subsection*{Uses}
\textcolor{red}{pygame.sprite.Sprite}

\subsection*{Syntax}
\subsubsection*{Exported Constants}
None
\subsubsection*{Exported Types}
Ammo = ?
\subsubsection*{Exported Access Programs}
\begin{tabular}{| l | l | l | p{5cm} |}
\hline
\textbf{Routine name} & \textbf{In} & \textbf{Out} & \textbf{Exceptions}\\
\hline
new Ammo & $\mathbb{R}$, $\mathbb{R}$, $\mathbb{R}$ & Ammo & IllegalArgumentException\\
\hline
\textcolor{red}{\st{setX}} & $\mathbb{R}$ &  & IllegalArgumentException\\
\hline
\textcolor{red}{\st{setY}} & $\mathbb{R}$ &  & IllegalArgumentException\\
\hline
\textcolor{red}{\st{getX }}&    & $\mathbb{R}$ & \\
\hline
\textcolor{red}{\st{getY}} &    & $\mathbb{R}$ & \\
\hline
\textcolor{red}{\st{reduce\_life}} & & & \\
\hline
\textcolor{red}{\st{isDead}} & & $\mathbb{B}$ & \\
\hline
\textcolor{red}{\st{move}} & & &\\
\hline
update & $\mathbb{Z}$ &&\\
\hline
getItemType &&$\mathbb{Z}$&\\
\hline
\end{tabular}

\subsection*{Semantics}
\subsubsection*{State Variables}
$\mathit{speed}$: $\mathbb{R}$\\
\textcolor{red}{\st{$\mathit{X}$: $\mathbb{R}$}}\\
\textcolor{red}{\st{$\mathit{Y}$: $\mathbb{R}$}}\\
\textcolor{red}{\st{$\mathit{X\_change}$: $\mathbb{R}$}}\\
\textcolor{red}{\st{$\mathit{Y\_change}$: $\mathbb{R}$}}\\
\textcolor{red}{\st{$\mathit{life}$: $\mathbb{N}$}}\\
itemType: $\mathbb{Z}$\\
$\mathit{image}$: \textit{.png file}\\
$\mathit{rect}$: \textit{image.get\_rect (This is the API of pygame library)}

\subsubsection*{State Invariant}
None
\subsubsection*{Assumptions}
None
\subsubsection*{Access Routine Semantics}
\noindent new Ammo($\mathit{x}, \mathit{y}, \mathit{s}$):
\begin{itemize}
\item transition: \\ speed := s\\ itemType := 3\\$image := \textit{corresponding image}$\\
 $rect := image.get\_rect(topleft\ =\ (x,\ y))$
\item output: $out := \mathit{self}$
\item exception: exc := $((\mathit{s} < 0) \vee (\mathit{x} < 0) \vee 
  (\mathit{y} < 0) \Rightarrow \text{IllegalArgumentException})$
\end{itemize}

\noindent \textcolor{red}{\st{setX($\mathit{x}$)}}:
\begin{itemize}
\item transition: $X := x$ 
\item output: none
\item exception: $((\mathit{x} < 0) \Rightarrow \text{IllegalArgumentException})$
\end{itemize}

\noindent \textcolor{red}{\st{setY($\mathit{y}$)}}:
\begin{itemize}
\item transition: $Y := y$ 
\item output: none
\item exception: $((\mathit{y} < 0) \Rightarrow \text{IllegalArgumentException})$
\end{itemize}

\noindent \textcolor{red}{\st{getX()}}:
\begin{itemize}
\item transition: none
\item output: $out := X$
\item exception: none
\end{itemize}

\noindent \textcolor{red}{\st{getY()}}:
\begin{itemize}
\item transition: none
\item output: $out := Y$
\item exception: none
\end{itemize}

\noindent \textcolor{red}{\st{reduce\_life()}}:
\begin{itemize}
\item transition: $life := life - 1$
\item output: none
\item exception: none
\end{itemize}

\noindent \textcolor{red}{\st{isDead()}}:
\begin{itemize}
\item transition: none
\item output: $life = 0$
\item exception: none
\end{itemize}

\noindent \textcolor{red}{\st{move()}}:
\begin{itemize}
\item transition: \\
$X := X + X\_change$\\
$X \leq 0 \Rightarrow (X\_change,\ Y  := speed,\ Y + Y\_change)$\\
$X \geq 736 \Rightarrow (X\_change,\ Y  := -1*speed,\ Y + Y\_change)$
\item output: none
\item exception: none
\end{itemize}


\noindent update(direction):
\begin{itemize}
\item transition: \\
rect.x := rect.x + (direction $\times$ speed)
\item output: none
\item exception: none
\end{itemize}

\noindent getItemType():
\begin{itemize}
\item transition: None
\item output: out := itemType
\item exception: none
\end{itemize}

\newpage
%%%%%%%%%%%%%%%%%%%%%%%%%%%%%%%%%%%%%%%%%%%%%%%%%%%%%%%%%%%%%%%%

%%%%%%%%%%%%%%%%%%%%%%%%%%%%Bomb%%%%%%%%%%%%%%%%%%%%%%%%%%%%%%%%
\section{Bomb Module}

\subsection*{Template Module}
Bomb

\subsection*{Uses}
\textcolor{red}{pygame.sprite.Sprite}

\subsection*{Syntax}
\subsubsection*{Exported Constants}
None
\subsubsection*{Exported Types}
Bomb = ?
\subsubsection*{Exported Access Programs}
\begin{tabular}{| l | l | l | p{5cm} |}
\hline
\textbf{Routine name} & \textbf{In} & \textbf{Out} & \textbf{Exceptions}\\
\hline
new Bomb & $\mathbb{R}$, $\mathbb{R}$, $\mathbb{R}$ & Bomb & IllegalArgumentException\\
\hline
\textcolor{red}{\st{setX}} & $\mathbb{R}$ &  & IllegalArgumentException\\
\hline
\textcolor{red}{\st{setY}} & $\mathbb{R}$ &  & IllegalArgumentException\\
\hline
\textcolor{red}{\st{getX }}&    & $\mathbb{R}$ & \\
\hline
\textcolor{red}{\st{getY}} &    & $\mathbb{R}$ & \\
\hline
\textcolor{red}{\st{reduce\_life}} & & & \\
\hline
\textcolor{red}{\st{isDead}} & & $\mathbb{B}$ & \\
\hline
\textcolor{red}{\st{move}} & & &\\
\hline
update & $\mathbb{Z}$ &&\\
\hline
getItemType &&$\mathbb{Z}$&\\
\hline
\end{tabular}

\subsection*{Semantics}
\subsubsection*{State Variables}
$\mathit{speed}$: $\mathbb{R}$\\
\textcolor{red}{\st{$\mathit{X}$: $\mathbb{R}$}}\\
\textcolor{red}{\st{$\mathit{Y}$: $\mathbb{R}$}}\\
\textcolor{red}{\st{$\mathit{X\_change}$: $\mathbb{R}$}}\\
\textcolor{red}{\st{$\mathit{Y\_change}$: $\mathbb{R}$}}\\
\textcolor{red}{\st{$\mathit{life}$: $\mathbb{N}$}}\\
itemType: $\mathbb{Z}$\\
$\mathit{image}$: \textit{.png file}\\
$\mathit{rect}$: \textit{image.get\_rect (This is the API of pygame library)}

\subsubsection*{State Invariant}
None
\subsubsection*{Assumptions}
None
\subsubsection*{Access Routine Semantics}
\noindent new Bomb($\mathit{x}, \mathit{y}, \mathit{s}$):
\begin{itemize}
\item transition: \\ speed := s\\ itemType := 4\\$image := \textit{corresponding image}$\\
 $rect := image.get\_rect(topleft\ =\ (x,\ y))$
\item output: $out := \mathit{self}$
\item exception: exc := $((\mathit{s} < 0) \vee (\mathit{x} < 0) \vee 
  (\mathit{y} < 0) \Rightarrow \text{IllegalArgumentException})$
\end{itemize}

\noindent \textcolor{red}{\st{setX($\mathit{x}$)}}:
\begin{itemize}
\item transition: $X := x$ 
\item output: none
\item exception: $((\mathit{x} < 0) \Rightarrow \text{IllegalArgumentException})$
\end{itemize}

\noindent \textcolor{red}{\st{setY($\mathit{y}$)}}:
\begin{itemize}
\item transition: $Y := y$ 
\item output: none
\item exception: $((\mathit{y} < 0) \Rightarrow \text{IllegalArgumentException})$
\end{itemize}

\noindent \textcolor{red}{\st{getX()}}:
\begin{itemize}
\item transition: none
\item output: $out := X$
\item exception: none
\end{itemize}

\noindent \textcolor{red}{\st{getY()}}:
\begin{itemize}
\item transition: none
\item output: $out := Y$
\item exception: none
\end{itemize}

\noindent \textcolor{red}{\st{reduce\_life()}}:
\begin{itemize}
\item transition: $life := life - 1$
\item output: none
\item exception: none
\end{itemize}

\noindent \textcolor{red}{\st{isDead()}}:
\begin{itemize}
\item transition: none
\item output: $life = 0$
\item exception: none
\end{itemize}

\noindent \textcolor{red}{\st{move()}}:
\begin{itemize}
\item transition: \\
$X := X + X\_change$\\
$X \leq 0 \Rightarrow (X\_change,\ Y  := speed,\ Y + Y\_change)$\\
$X \geq 736 \Rightarrow (X\_change,\ Y  := -1*speed,\ Y + Y\_change)$
\item output: none
\item exception: none
\end{itemize}


\noindent update(direction):
\begin{itemize}
\item transition: \\
rect.x := rect.x + (direction $\times$ speed)
\item output: none
\item exception: none
\end{itemize}

\noindent getItemType():
\begin{itemize}
\item transition: None
\item output: out := itemType
\item exception: none
\end{itemize}

\newpage

%%%%%%%%%%%%%%%%%%%%%%%%%%%%%%%%%%%%%%%%%%%%%%%%%%%%%%%%%%%%%%%%

%%%%%%%%%%%%%%%%%%%%%%%%%%%%%%Heart%%%%%%%%%%%%%%%%%%%%%%%%%%%%%
\section{Heart Module}

\subsection*{Template Module}
Heart

\subsection*{Uses}
\textcolor{red}{pygame.sprite.Sprite}

\subsection*{Syntax}
\subsubsection*{Exported Constants}
None
\subsubsection*{Exported Types}
Heart = ?
\subsubsection*{Exported Access Programs}
\begin{tabular}{| l | l | l | p{5cm} |}
\hline
\textbf{Routine name} & \textbf{In} & \textbf{Out} & \textbf{Exceptions}\\
\hline
new Heart & $\mathbb{R}$, $\mathbb{R}$, $\mathbb{R}$ & Heart & IllegalArgumentException\\
\hline
\textcolor{red}{\st{setX}} & $\mathbb{R}$ &  & IllegalArgumentException\\
\hline
\textcolor{red}{\st{setY}} & $\mathbb{R}$ &  & IllegalArgumentException\\
\hline
\textcolor{red}{\st{getX }}&    & $\mathbb{R}$ & \\
\hline
\textcolor{red}{\st{getY}} &    & $\mathbb{R}$ & \\
\hline
\textcolor{red}{\st{reduce\_life}} & & & \\
\hline
\textcolor{red}{\st{isDead}} & & $\mathbb{B}$ & \\
\hline
\textcolor{red}{\st{move}} & & &\\
\hline
update & $\mathbb{Z}$ &&\\
\hline
getItemType &&$\mathbb{Z}$&\\
\hline
\end{tabular}

\subsection*{Semantics}
\subsubsection*{State Variables}
$\mathit{speed}$: $\mathbb{R}$\\
\textcolor{red}{\st{$\mathit{X}$: $\mathbb{R}$}}\\
\textcolor{red}{\st{$\mathit{Y}$: $\mathbb{R}$}}\\
\textcolor{red}{\st{$\mathit{X\_change}$: $\mathbb{R}$}}\\
\textcolor{red}{\st{$\mathit{Y\_change}$: $\mathbb{R}$}}\\
\textcolor{red}{\st{$\mathit{life}$: $\mathbb{N}$}}\\
itemType: $\mathbb{Z}$\\
$\mathit{image}$: \textit{.png file}\\
$\mathit{rect}$: \textit{image.get\_rect (This is the API of pygame library)}

\subsubsection*{State Invariant}
None
\subsubsection*{Assumptions}
None
\subsubsection*{Access Routine Semantics}
\noindent new Heart($\mathit{x}, \mathit{y}, \mathit{s}$):
\begin{itemize}
\item transition: \\ speed := s\\ itemType := 2\\$image := \textit{corresponding image}$\\
 $rect := image.get\_rect(topleft\ =\ (x,\ y))$
\item output: $out := \mathit{self}$
\item exception: exc := $((\mathit{s} < 0) \vee (\mathit{x} < 0) \vee 
  (\mathit{y} < 0) \Rightarrow \text{IllegalArgumentException})$
\end{itemize}

\noindent \textcolor{red}{\st{setX($\mathit{x}$)}}:
\begin{itemize}
\item transition: $X := x$ 
\item output: none
\item exception: $((\mathit{x} < 0) \Rightarrow \text{IllegalArgumentException})$
\end{itemize}

\noindent \textcolor{red}{\st{setY($\mathit{y}$)}}:
\begin{itemize}
\item transition: $Y := y$ 
\item output: none
\item exception: $((\mathit{y} < 0) \Rightarrow \text{IllegalArgumentException})$
\end{itemize}

\noindent \textcolor{red}{\st{getX()}}:
\begin{itemize}
\item transition: none
\item output: $out := X$
\item exception: none
\end{itemize}

\noindent \textcolor{red}{\st{getY()}}:
\begin{itemize}
\item transition: none
\item output: $out := Y$
\item exception: none
\end{itemize}

\noindent \textcolor{red}{\st{reduce\_life()}}:
\begin{itemize}
\item transition: $life := life - 1$
\item output: none
\item exception: none
\end{itemize}

\noindent \textcolor{red}{\st{isDead()}}:
\begin{itemize}
\item transition: none
\item output: $life = 0$
\item exception: none
\end{itemize}

\noindent \textcolor{red}{\st{move()}}:
\begin{itemize}
\item transition: \\
$X := X + X\_change$\\
$X \leq 0 \Rightarrow (X\_change,\ Y  := speed,\ Y + Y\_change)$\\
$X \geq 736 \Rightarrow (X\_change,\ Y  := -1*speed,\ Y + Y\_change)$
\item output: none
\item exception: none
\end{itemize}


\noindent update(direction):
\begin{itemize}
\item transition: \\
rect.x := rect.x + (direction $\times$ speed)
\item output: none
\item exception: none
\end{itemize}

\noindent getItemType():
\begin{itemize}
\item transition: None
\item output: out := itemType
\item exception: none
\end{itemize}

\newpage

%%%%%%%%%%%%%%%%%%%%%%%%%%%%%%%%%%%%%%%%%%%%%%%%%%%%%%%%%%%%%%%%

%%%%%%%%%%%%%%%%%%%%%%%%%%%%%%%CollisionDetection%%%%%%%%%%%%%%%
\section{CollisionDectection Module \textcolor{red}{This module has been deleted}}

\subsection*{Service Module}
Service

\subsection*{Uses}
None

\subsection*{Syntax}
\subsubsection*{Exported Constants}
None
\subsubsection*{Exported Types}
None
\subsubsection*{Exported Access Programs}
\begin{tabular}{| l | l | l | p{5cm} |}
\hline
\textbf{Routine name} & \textbf{In} & \textbf{Out} & \textbf{Exceptions}\\
\hline
isCollided & $\mathbb{R}$, $\mathbb{R}$, $\mathbb{R}$, $\mathbb{R}$ & $\mathbb{B}$ & \\
\hline
\end{tabular}

\subsection*{Semantics}
\subsubsection*{State Variables}
None
\subsubsection*{State Invariant}
None
\subsubsection*{Assumptions}
None
\subsubsection*{Access Routine Semantics}
\noindent isCollided ($x_1, x_2, y_1, y_2$):
\begin{itemize}
\item transition: none
\item output: $distance(x_1, x_2, y_1, y_2) \leq 27$
\item exception: exc := none
\end{itemize}

\subsection*{Local Function}
distance: [$\mathbb{R}$, $\mathbb{R}$, $\mathbb{R}$, $\mathbb{R}$] $\Rightarrow$ $\mathbb{R}$\\
distance($x_1, x_2, y_1, y_2$) $\equiv$ $\sqrt{(x_1 - x_2)^2 + (y_1 - y_2)^2}$
\newpage
%%%%%%%%%%%%%%%%%%%%%%%%%%%%%%%%%%%%%%%%%%%%%%%%%%%%%%%%%%%%%%%%

%%%%%%%%%%%%%%%%%%%%%%%%%%%%%%MonsterMatrix%%%%%%%%%%%%%%%%%%%%%
\section{MonsterMatrix Module}

\subsection*{Template Module}
MonsterMatrix

\subsection*{Uses}
Monster, Ammo, Heart, Bomb, \textcolor{red}{pygame.sprite.Sprite}
\subsection*{Syntax}
\subsubsection*{Exported Constants}
None
\subsubsection*{Exported Types}
MonsterMatrix = ?
\subsubsection*{Exported Access Programs}
\begin{tabular}{| l | l | l | p{5cm} |}
\hline
\textbf{Routine name} & \textbf{In} & \textbf{Out} & \textbf{Exceptions}\\
\hline
new MonsterMatrix & $\mathbb{Z}$, $\mathbb{R}$, ($\mathbb{N}, \mathbb{N}$) & MonsterMatrix & IllegalArgumentException\\
\hline
\textcolor{red}{\st{getMatrix}} && seq of seq [Monster, Ammo, Heart, Bomb] &\\
\hline
\textcolor{red}{\st{move}} &&&\\
\hline
shoot &&&\\
\hline
round1 &&&\\
\hline
round2 &&&\\
\hline
round3 &&&\\
\hline
round4 &&&\\
\hline
round5 &&&\\
\hline
getMonstersGroup &&pygame sprite group&\\
\hline
boundaryDetection &&&\\
\hline
move\_down  &&&\\
\hline
getBulletsGroup &&pygame sprite group&\\
\hline
update &&&\\
\hline
\end{tabular}

\subsection*{Semantics}
\subsubsection*{State Variables}
$\mathit{speed}$: $\mathbb{R}$\\
\textcolor{red}{\st{$\mathit{M}$: $seq\ of\ seq[Ammo\ Monster\ Heart\ Bomb]$}}\\
monsters\_group : pygame sprite group\\
direction : $\mathbb{Z}$\\
screen\_size\_info: ($\mathbb{R}, \mathbb{R}$)\\
bullets\_group: pygame sprite group
\subsubsection*{State Invariant}
None
\subsubsection*{Assumptions}
None
\subsubsection*{Access Routine Semantics}
\noindent new MonsterMatrix($\mathit{round}, \mathit{s}$):
\begin{itemize}
\item transition: \\$\mathit{speed} := s$\\
$\mathit{round} = 1 \Rightarrow M := m1(with\ monsters\ randomly\ replaced\ by\ Ammo,\ Bomb,\ Heart)$\\
$\mathit{round} = 2 \Rightarrow M := m2(with\ monsters\ randomly\ replaced\ by\ Ammo,\ Bomb,\ Heart)$\\
$\mathit{round} = 3 \Rightarrow M := m3(with\ monsters\ randomly\ replaced\ by\ Ammo,\ Bomb,\ Heart)$\\
$\mathit{round} = 4 \Rightarrow M := m4(with\ monsters\ randomly\ replaced\ by\ Ammo,\ Bomb,\ Heart)$\\
$\mathit{round} = 5 \Rightarrow M := m5(with\ monsters\ randomly\ replaced\ by\ Ammo,\ Bomb,\ Heart)$
\item output: $out := \mathit{self}$
\item exception: exc := $(\lnot (0 \leq \mathit{round} \leq 5) \vee (\mathit{s} < 0)) \Rightarrow \text{IllegalArgumentException})$
\end{itemize}

\noindent round1():
\begin{itemize}
\item transition: add m1 to monsters\_group with game items randomly replaced
\item output: none
\item exception: none
\end{itemize}

\noindent round2():
\begin{itemize}
\item transition: add m2 to monsters\_group with game items randomly replaced
\item output: none
\item exception: none
\end{itemize}


\noindent round3():
\begin{itemize}
\item transition: add m3 to monsters\_group with game items randomly replaced
\item output: none
\item exception: none
\end{itemize}


\noindent round4():
\begin{itemize}
\item transition: add m4 to monsters\_group with game items randomly replaced
\item output: none
\item exception: none
\end{itemize}

\noindent round5():
\begin{itemize}
\item transition: add m5 to monsters\_group with game items randomly replaced
\item output: none
\item exception: none
\end{itemize}

\noindent getMonstersGroup():
\begin{itemize}
\item transition: none
\item output: out := monsters\_group
\item exception: none
\end{itemize}

\noindent boundaryDetection():
\begin{itemize}
\item transition: if any of the monster in monsters\_group touches the edge
of the screen, move the whole monster matrix down.
\item output: none
\item exception: none
\end{itemize}

\noindent move\_down():
\begin{itemize}
\item transition: $\forall\ gameItem \in monsters\_group : gameItem.rect.y\ +=\ 2$
\item output: none
\item exception: none
\end{itemize}

\noindent \textcolor{red}{\st{\noindent getMatrix()}  Method removed}:
\begin{itemize}
\item transition: none
\item output: $out := M$
\item exception: none
\end{itemize}

\noindent \textcolor{red}{\st{move()} Method removed}:
\begin{itemize}
\item transition: Monster Matrix moves in direction east $\rightarrow$ south $\rightarrow$ west
\item output: none
\item exception: none
\end{itemize}

\noindent shoot():
\begin{itemize}
\item transition: Monsters $M$ shoot bullets randomly and add bullets object to
bullets\_group.
\item output: none
\item exception: none
\end{itemize}


\noindent getBulletsGroup():
\begin{itemize}
\item transition: none
\item output: out := bullets\_group
\item exception: none
\end{itemize}

\noindent update():
\begin{itemize}
\item transition: \\$\forall\ gameItem \in monsters\_group : gameItem.update()$\\
boundaryDetection()\\
bullets\_group.update()
\item output: none
\item exception: none
\end{itemize}

\subsection*{Local Types}
m1 $\equiv$ 
$
\begin{bmatrix}
GM & GM & GM & GM & GM & GM & GM & GM & GM & GM \\
GM & GM & GM & GM & GM & GM & GM & GM & GM & GM \\
GM & GM & GM & GM & GM & GM & GM & GM & GM & GM \\
GM & GM & GM & GM & GM & GM & GM & GM & GM & GM \\
GM & GM & GM & GM & GM & GM & GM & GM & GM & GM 
\end{bmatrix}
$

\vspace{1cm}
\noindent 
m2 $\equiv$ 
$
\begin{bmatrix}
BM & BM & BM & BM & BM & BM & BM & BM & BM & BM \\
BM & BM & BM & BM & BM & BM & BM & BM & BM & BM \\
GM & GM & GM & GM & GM & GM & GM & GM & GM & GM \\
GM & GM & GM & GM & GM & GM & GM & GM & GM & GM \\
GM & GM & GM & GM & GM & GM & GM & GM & GM & GM 
\end{bmatrix}
$

\vspace{1cm}
\noindent 
m3 $\equiv$ 
$
\begin{bmatrix}
BM & BM & BM & BM & BM & BM & BM & BM & BM & BM \\
BM & BM & BM & BM & BM & BM & BM & BM & BM & BM \\
BM & BM & BM & BM & BM & BM & BM & BM & BM & BM \\
BM & BM & BM & BM & BM & BM & BM & BM & BM & BM \\
BM & BM & BM & BM & BM & BM & BM & BM & BM & BM 
\end{bmatrix}
$

\vspace{1cm}
\noindent 
m4 $\equiv$ 
$
\begin{bmatrix}
PM & PM & PM & PM & PM & PM & PM & PM & PM & PM \\
BM & BM & BM & BM & BM & BM & BM & BM & BM & BM \\
BM & BM & BM & BM & BM & BM & BM & BM & BM & BM \\
BM & BM & BM & BM & BM & BM & BM & BM & BM & BM \\
GM & GM & GM & GM & GM & GM & GM & GM & GM & GM 
\end{bmatrix}
$

\vspace{1cm}
\noindent 
m5 $\equiv$ 
$
\begin{bmatrix}
PM & PM & PM & PM & PM & PM & PM & PM & PM & PM \\
PM & PM & PM & PM & PM & PM & PM & PM & PM & PM \\
PM & PM & PM & PM & PM & PM & PM & PM & PM & PM \\
PM & PM & PM & PM & PM & PM & PM & PM & PM & PM \\
PM & PM & PM & PM & PM & PM & PM & PM & PM & PM 
\end{bmatrix}
$
\\

\noindent GM means green monster.\\ BM means blue monster.\\ PM means pink monster.
\newpage
%%%%%%%%%%%%%%%%%%%%%%%%%%%%%%%%%%%%%%%%%%%%%%%%%%%%%%%%%%%%%%%%


%%%%%%%%%%%%%%%%%%%%% MonsterDisplay %%%%%%%%%%%%%%%%%%%%%%%%%%%
\section{MonsterDisplay Module \textcolor{red}{This module has been deleted}}

\subsection*{UserInterface Module}
MonsterDisplay
\subsection*{Uses}
Monster, MonsterColor

\subsection*{Syntax}
\subsubsection*{Exported Constants}
N/A
\subsubsection*{Exported Types}
MonsterDisplay = ?

\subsubsection*{Exported Access Programs}
\begin{tabular}{| l | l | l | l |}
\hline
\textbf{Routine name} & \textbf{In} & \textbf{Out} & \textbf{Exceptions}\\
\hline
new MonsterDisplay&pygame window object, MonsterColor&     MonsterDisplay  & \\
\hline
show       &	$\mathbb{N}$,	$\mathbb{N}$, Boolean      &           &          \\
\hline
\end{tabular}

\subsection*{Semantics}
\subsubsection*{Environment Variables}
screen: It is the game screen that is manipulated by the following functions to alter display. This is update by a precise frame rate to depict various game objects on the game screen.

\subsubsection*{State Variables}
$SCREEN: pygame\ window\ object$\\
$img: \text{Monster Picture}$
\subsubsection*{State Invariant}
N/A
\subsubsection*{Assumptions}
N/A
\subsubsection*{Access Routine Semantics}

new MonsterDisplay($screen, monster\_color$)
\begin{itemize}
    \item transition: \\$SCREEN := screen$
    \\$monster\_color = MonsterColor.GREEN \Rightarrow img := \text{green monster picture}$
    \\$monster\_color = MonsterColor.BLUE \Rightarrow img := \text{blue monster picture}$
    \\$monster\_color = MonsterColor.PINK \Rightarrow img := \text{pink monster picture}$
    \item output: $out := \mathit{self}$
    \item exception: None
\end{itemize}

\noindent show($x, y, isDead$):
\begin{itemize}
    \item transition:
    $isDead \Rightarrow Display\ img\ at\ (x, y)$
    \item output: none
    \item input definitions: $x$ and $y$ represent the monster display coordinate. $isDead$ is used to clarify whether the monster is killed.
    \item exception: None
\end{itemize}
\newpage
%%%%%%%%%%%%%%%%%%%%%%%%%%%%%%%%%%%%%%%%%%%%%%%%%%%%%%%%%%%%%%%%

%%%%%%%%%%%%%%%%%%%%%%%SpaceShipDisplay%%%%%%%%%%%%%%%%%%%%%%%%%
\section{SpaceShipDisplay Module}

\subsection*{UserInterface Module}
SpaceShipDisplay
\subsection*{Uses}
SpaceShip, \textcolor{red}{pygame.sprite.Sprite}

\subsection*{Syntax}
\subsubsection*{Exported Constants}
N/A
\subsubsection*{Exported Types}
SpaceShipDisplay = ?

{\small
\subsubsection*{Exported Access Programs}
\begin{tabular}{| l | l | l | l |}
\hline
\textbf{Routine name} & \textbf{In} & \textbf{Out} & \textbf{Exceptions}\\
\hline
new SpaceShipDisplay       &pygame window object, \textcolor{red}{\st{$\mathbb{N}$}}, 
pygame sprite group     &     SpaceShipDisplay      &           \\
\hline
show       &	\textcolor{red}{\st{$\mathbb{N}$,	$\mathbb{N}$ }}    &           &          \\
\hline
\end{tabular}
}

\subsection*{Semantics}
\subsubsection*{Environment Variables}
screen: It is the game screen that is manipulated by the following functions to alter display. This is update by a precise frame rate to depict various game objects on the game screen.
\subsubsection*{State Variables}
N/A
\subsubsection*{State Invariant}
$SCREEN: pygame\ window\ object$\\
\textcolor{red}{\st{$img: \text{SpaceShip Picture}$}}\\
space\_ship\_group: pygame sprite group
\subsubsection*{Assumptions}
N/A
\subsubsection*{Access Routine Semantics}
new SpaceShipDisplay($screen$, \textcolor{red}{\st{$spaceship\_number$}}, group)
\begin{itemize}
    \item transition: \\$SCREEN := screen$
    \\\textcolor{red}{\st{$space\_number = 1 \Rightarrow img := \text{spaceship1 picture}$}}
    \\\textcolor{red}{\st{$space\_number = 2 \Rightarrow img := \text{spaceship2 picture}$}}\\
    space\_ship\_group := group
    \item output: $out := self$
    \item exception: None
\end{itemize}

\noindent show(\textcolor{red}{\st{$x, y$}}):
\begin{itemize}
    \item transition: space\_ship\_group.draw(SCREEN) \# This is pygame API
    \item output: none
    \item exception: None
\end{itemize}
\newpage
%%%%%%%%%%%%%%%%%%%%%%%%%%%%%%%%%%%%%%%%%%%%%%%%%%%%%%%%%%%%%%%%

%%%%%%%%%%%%%%%%%%%%%%%%%BulletDisplay%%%%%%%%%%%%%%%%%%%%%%%%%%%
\section{BulletDisplay Module}

\subsection*{UserInterface Module}
BulletDisplay
\subsection*{Uses}
BulletState, Bullet, \textcolor{red}{pygame.sprite.Sprite}

\subsection*{Syntax}
\subsubsection*{Exported Constants}
N/A
\subsubsection*{Exported Types}
BulletDisplay = ?

\subsubsection*{Exported Access Programs}

\begin{tabular}{| l | l | l | l |}
\hline
\textbf{Routine name} & \textbf{In} & \textbf{Out} & \textbf{Exceptions}\\
\hline
new BulletDisplay       &pygame window object, pygame sprite group&     BulletDisplay        &           \\
\hline
show       &	\textcolor{red}{\st{$\mathbb{N}$, $\mathbb{N}$, $BulletState$   }}  &           &          \\
\hline
\end{tabular}

\subsection*{Semantics}
\subsubsection*{Environment Variables}
screen: It is the game screen that is manipulated by the following functions to alter display. This is update by a precise frame rate to depict various game objects on the game screen.
\subsubsection*{State Variables}
$SCREEN: pygame\ window\ object$\\
\textcolor{red}{\st{$img: \text{Bullet Picture}$}}\\
bullets\_group: pygame sprite group
\subsubsection*{State Invariant}
N/A
\subsubsection*{Assumptions}
N/A
\subsubsection*{Access Routine Semantics}

new BulletDisplay($screen$, \textcolor{red}{bullets})
\begin{itemize}
     \item transition: \\$SCREEN := screen$
    \\\textcolor{red}{\st{$img := \text{bullet picture}$}}\\
    \textcolor{red}{bullets\_group := bullets}
    \item output: $out := self$
    \item exception: None
\end{itemize}

\noindent show(\textcolor{red}{\st{$x, y, state$}}):
\begin{itemize}
    \item transition: \textcolor{red}{\st{$state = BulletState.FIRE \Rightarrow \text{Display}\ img\ at\ (x, y)$}}\\
    \textcolor{red}{bullets\_group.draw(SCREEN) \# This is pygame API}
     \item output: none
    \item exception: None
\end{itemize}
\newpage
%%%%%%%%%%%%%%%%%%%%%%%%%%%%%%%%%%%%%%%%%%%%%%%%%%%%%%%%%%%%%%%%%

%%%%%%%%%%%%%%%%%%%%%%%%ScoreDisplay%%%%%%%%%%%%%%%%%%%%%%%%%%%%%
\newpage
\section{ScoreDisplay Module}

\subsection*{UserInterface Module}
ScoreDisplay
\subsection*{Uses}
Score
\subsection*{Syntax}
\subsubsection*{Exported Constants}
N/A
\subsubsection*{Exported Types}
ScoreDisplay = ?
\subsubsection*{Exported Access Programs}
\begin{tabular}{| l | l | l | l |}
\hline
\textbf{Routine name} & \textbf{In} & \textbf{Out} & \textbf{Exceptions}\\
\hline
new ScoreDisplay       &pygame window object&     ScoreDisplay        &           \\
\hline
show       &	$\mathbb{N}$, $\mathbb{N}$, $\mathbb{N}$     &           &          \\
\hline
\end{tabular}

\subsection*{Semantics}
\subsubsection*{Environment Variables}
screen: It is the game screen that is manipulated by the following functions to alter display. This is update by a precise frame rate to depict various game objects on the game screen.

\subsubsection*{State Variables}
$SCREEN: pygame\ window\ object$
\subsubsection*{State Invariant}
N/A
\subsubsection*{Assumptions}
N/A
\subsubsection*{Access Routine Semantics}

new ScoreDisplay($screen$)
\begin{itemize}
    \item transition: $SCREEN := screen$
    \item output: $out := self$
    \item exception: None
\end{itemize}

\noindent show($x, y, score$):
\begin{itemize}
    \item transition: Display $score$ at $(x, y)$.
    \item input definitions: $x$ and $y$ represent the score's coordinate. $score$ represent the total scores of player(s).
    \item output: none
    \item exception: None
\end{itemize}
\newpage
%%%%%%%%%%%%%%%%%%%%%%%%%%%%%%%%%%%%%%%%%%%%%%%%%%%%%%%%%%%%%%%%%

%%%%%%%%%%%%%%%%%%%%%%ObstacleDisplay%%%%%%%%%%%%%%%%%%%%%%%%%%%
\section{ObstaclesDisplay Module}

\subsection*{UserInterface Module}
ObstaclesDisplay

\subsection*{Uses}
Obstacle \textcolor{red}{pygame.sprite.Sprite}

\subsection*{Syntax}
\subsubsection*{Exported Constants}
N/A
\subsubsection*{Exported Types}
ObstaclesDisplay = ?

\subsubsection*{Exported Access Programs}
\begin{tabular}{| l | l | l | l |}
\hline
\textbf{Routine name} & \textbf{In} & \textbf{Out} & \textbf{Exceptions}\\
\hline
new ObstaclesDisplay       &pygame window object, pygame sprite group&     ObstaclesDisplay        &           \\
\hline
show       &	\textcolor{red}{\st{$\mathbb{N}$, $\mathbb{N}$, $\mathbb{B}$ }}    &           &          \\
\hline
\end{tabular}

\subsection*{Semantics}
\subsubsection*{Environment Variables}
screen: It is the game screen that is manipulated by the following functions to alter display. This is update by a precise frame rate to depict various game objects on the game screen.
\subsubsection*{State Variables}
$SCREEN: pygame\ window\ object$\\
\textcolor{red}{\st{$img: \text{Obstacle Picture}$}}\\
blocks\_group: pygame sprite group
\subsubsection*{State Invariant}
N/A
\subsubsection*{Assumptions}
N/A
\subsubsection*{Access Routine Semantics}

new ObstaclesDisplay($screen$, \textcolor{red}{blocks})
\begin{itemize}
    \item transition: \\
    $SCREEN := screen$\\
    \textcolor{red}{\st{$img := \text{Obstacle Picture}$}}\\
    \textcolor{red}{blocks\_group := blocks}
    \item output: $out := self$
    \item exception: None
\end{itemize}

\noindent show(\textcolor{red}{\st{$x, y, isDestroy$}}):
\begin{itemize}
    \item transition: \textcolor{red}{\st{$\lnot isDestroy \Rightarrow Display\ img\ at\ (x, y)$}}\\
    \textcolor{red}{blocks\_group.draw(SCREEN) \# This is pygame API}
    
    \item output: none
    \item exception: None
\end{itemize}
\newpage
%%%%%%%%%%%%%%%%%%%%%%%%%%%%%%%%%%%%%%%%%%%%%%%%%%%%%%%%%%%%%%%%

%%%%%%%%%%%%%%%%%%%%%%%%%Ammo Display%%%%%%%%%%%%%%%%%%%%%%%%%%%
\section{AmmoDisplay Module \textcolor{red}{This module has been deleted}}

\subsection*{UserInterface Module}
AmmoDisplay

\subsection*{Uses}
Ammo

\subsection*{Syntax}
\subsubsection*{Exported Constants}
N/A
\subsubsection*{Exported Types}
AmmoDisplay = ?
\subsubsection*{Exported Access Programs}

\begin{tabular}{| l | l | l | l |}
\hline
\textbf{Routine name} & \textbf{In} & \textbf{Out} & \textbf{Exceptions}\\
\hline
new AmmoDisplay       &pygame window object&     AmmoDisplay        &           \\
\hline
show       &	 $\mathbb{N}$, $\mathbb{N}$, $\mathbb{B}$    &           &          \\
\hline
\end{tabular}

\subsection*{Semantics}
\subsubsection*{Environment Variables}
screen: It is the game screen that is manipulated by the following functions to alter display. This is update by a precise frame rate to depict various game objects on the game screen.
\subsubsection*{State Variables}
$SCREEN : \text{pygame window object}$\\
$img : \text{Picture of Ammo}$
\subsubsection*{State Invariant}
N/A
\subsubsection*{Assumptions}
N/A
\subsubsection*{Access Routine Semantics}

new AmmoDisplay($screen$)
\begin{itemize}
    \item transition:\\
    $SCRREN := screen$\\
    $img := Ammo\ Picture$
    \item output: $out := self$
    \item exception: None
\end{itemize}

\noindent show(x, y, isShot):
\begin{itemize}
    \item transition: $\lnot isShot \Rightarrow Display\ img\ at\ (x, y)$
    \item input definitions: x, y represent the coordinate of ammo picture
    to be displayed. $isShot$ represents the state of ammo, it is True if the ammo is shoot.
    \item output: none
    \item exception: None
\end{itemize}
\newpage
%%%%%%%%%%%%%%%%%%%%%%%%%%%%%%%%%%%%%%%%%%%%%%%%%%%%%%%%%%%%%%%%

%%%%%%%%%%%%%%%%%%%%%%%%%Heart Display%%%%%%%%%%%%%%%%%%%%%%%%%%%
\section{HeartDisplay Module \textcolor{red}{This module has been deleted}}

\subsection*{UserInterface Module}
HeartDisplay

\subsection*{Uses}
Heart

\subsection*{Syntax}
\subsubsection*{Exported Constants}
N/A
\subsubsection*{Exported Types}
HeartDisplay = ?
\subsubsection*{Exported Access Programs}

\begin{tabular}{| l | l | l | l |}
\hline
\textbf{Routine name} & \textbf{In} & \textbf{Out} & \textbf{Exceptions}\\
\hline
new HeartDisplay       &pygame window object&     HeartDisplay        &           \\
\hline
show       &	 $\mathbb{N}$, $\mathbb{N}$, $\mathbb{B}$    &           &          \\
\hline
\end{tabular}

\subsection*{Semantics}
\subsubsection*{Environment Variables}
screen: It is the game screen that is manipulated by the following functions to alter display. This is update by a precise frame rate to depict various game objects on the game screen.
\subsubsection*{State Variables}
$SCREEN : \text{pygame window object}$\\
$img : \text{Picture of Heart}$
\subsubsection*{State Invariant}
N/A
\subsubsection*{Assumptions}
N/A
\subsubsection*{Access Routine Semantics}

new HeartDisplay($screen$)
\begin{itemize}
    \item transition:\\
    $SCRREN := screen$\\
    $img := Heart\ Picture$
    \item output: $out := self$
    \item exception: None
\end{itemize}

\noindent show(x, y, isShot):
\begin{itemize}
    \item transition: $\lnot isShot \Rightarrow Display\ img\ at\ (x, y)$
    \item input definitions: x, y represent the coordinate of Heart
    picture to be displayed. $isShot$ represents the state of Heart, it is True if the Heart is shoot.
    \item output: none
    \item exception: None
\end{itemize}
\newpage
%%%%%%%%%%%%%%%%%%%%%%%%%%%%%%%%%%%%%%%%%%%%%%%%%%%%%%%%%%%%%%%%

%%%%%%%%%%%%%%%%%%%%%%%Bomb%%%%%%%%%%%%%%%%%%%%%%%%%%%%%%%%%%%%%
\section{BombDisplay Module \textcolor{red}{This module has been deleted}}

\subsection*{UserInterface Module}
BombDisplay

\subsection*{Uses}
Bomb

\subsection*{Syntax}
\subsubsection*{Exported Constants}
N/A
\subsubsection*{Exported Types}
BombDisplay = ?
\subsubsection*{Exported Access Programs}

\begin{tabular}{| l | l | l | l |}
\hline
\textbf{Routine name} & \textbf{In} & \textbf{Out} & \textbf{Exceptions}\\
\hline
new BombDisplay       &pygame window object&     BombDisplay        &           \\
\hline
show       &	 $\mathbb{N}$, $\mathbb{N}$, $\mathbb{B}$    &           &          \\
\hline
\end{tabular}

\subsection*{Semantics}
\subsubsection*{Environment Variables}
screen: It is the game screen that is manipulated by the following functions to alter display. This is update by a precise frame rate to depict various game objects on the game screen.
\subsubsection*{State Variables}
$SCREEN : \text{pygame window object}$\\
$img : \text{Picture of Bomb}$
\subsubsection*{State Invariant}
N/A
\subsubsection*{Assumptions}
N/A
\subsubsection*{Access Routine Semantics}

new BombDisplay($screen$)
\begin{itemize}
    \item transition:\\
    $SCRREN := screen$\\
    $img := Bomb\ Picture$
    \item output: $out := self$
    \item exception: None
\end{itemize}

\noindent show(x, y, isShot):
\begin{itemize}
    \item transition: $\lnot isShot \Rightarrow Display\ img\ at\ (x, y)$
    \item input definitions: x, y represent the coordinate of Bomb
    picture
     to be displayed. $isShot$ represents the state of Bomb, it is True if the Bomb is shoot.
    \item output: none
    \item exception: None
\end{itemize}
\newpage
%%%%%%%%%%%%%%%%%%%%%%%%%%%%%%%%%%%%%%%%%%%%%%%%%%%%%%%%%%%%%%%%

%%%%%%%%%%%%%%%%%%%%%%%%%MonsterMatrixDisplay%%%%%%%%%%%%%%%%%%%
\section{MonsterMatrixDisplay Module}

\subsection*{UserInterface Module}
MonsterMatrixDisplay

\subsection*{Uses}
\textcolor{red}{\st{BombDisplay, MonsterDisplay, AmmoDisplay, HeartDisplay,}} MonsterMatrix, \textcolor{red}{pygame.sprite.Sprite}

\subsection*{Syntax}
\subsubsection*{Exported Constants}
N/A
\subsubsection*{Exported Types}
MonsterMatrixDisplay = ?
\subsubsection*{Exported Access Programs}
\begin{tabular}{| l | l | l | l |}
\hline
\textbf{Routine name} & \textbf{In} & \textbf{Out} & \textbf{Exceptions}\\
\hline
new MonsterMatrixDisplay&pygame window object, pygame sprite group &MonsterMatrixDisplay&\\
\hline
show       &	\textcolor{red}{\st{ $MonsterMatrix$ }} &           &          \\
\hline
\end{tabular}

\subsection*{Semantics}
\subsubsection*{Environment Variables}
screen: It is the game screen that is manipulated by the following functions to alter display. This is update by a precise frame rate to depict various game objects on the game screen.
\subsubsection*{State Variables}
$SCREEN : \text{pygame window object}$\\
monsters\_group : pygame sprite group
\subsubsection*{State Invariant}
N/A
\subsubsection*{Assumptions}
N/A
\subsubsection*{Access Routine Semantics}
new MonsterMatrixDisplay($screen$, \textcolor{red}{monsters})
\begin{itemize}
    \item transition:\\ $SCREEN := screen$ \\ \textcolor{red}{monsters\_group := monsters}
    \item output: $out := self$
    \item exception: None
\end{itemize}

\noindent show(\textcolor{red}{\st{$M$}}):
\begin{itemize}
    \item transition: \textcolor{red}{\st{$\forall object \in M\ |\ object.show(x,\ y,\ isDead/isShot)$\\
    $M$ here could be $Monster\ Ammo\ Heart\ Bomb$}}\\
    \textcolor{red}{monsters\_group.draw{SCREEN} \# This is pygame API}
    \item output: none
    \item exception: None
\end{itemize}
\newpage
%%%%%%%%%%%%%%%%%%%%%%%%%%%%%%%%%%%%%%%%%%%%%%%%%%%%%%%%%%%%%%%%

%%%%%%%%%%%%%%%%%%%%%%%%%%%%%%%SetUpDisplay%%%%%%%%%%%%%%%%%%%%%
\section{\textcolor{red}{ WindowSetUp} Module}

\subsection*{UserInterface Module}
\textcolor{red}{\st{SetUpDisplay} WindowSetUp }
\subsection*{Uses}
None

\subsection*{Syntax}
\subsubsection*{Exported Constants}
N/A
\subsubsection*{Exported Types}
\textcolor{red}{\st{SetUpDisplay} WindowSetUp } = ?

\subsubsection*{Exported Access Programs}
\begin{tabular}{| l | l | l | l |}
\hline
\textbf{Routine name} & \textbf{In} & \textbf{Out} & \textbf{Exceptions}\\
\hline
new \textcolor{red}{\st{SetUpDisplay} WindowSetUp }       &      \textcolor{red}{$\mathbb{Z}, \mathbb{Z}$}     &     \textcolor{red}{\st{SetUpDisplay} WindowSetUp }       &           \\
\hline
\textcolor{red}{\st{show  } Method removed}     &	   &           &          \\
\hline
getScreen      &	   &pygame window  object&          \\
\hline
setTitle &&&\\
\hline
setIcon &&&\\
\hline
setBackground &&&\\
\hline
\end{tabular}

\subsection*{Semantics}
\subsubsection*{Environment Variables}
screen: It is the game screen that is manipulated by the following functions to alter display. This is update by a precise frame rate to depict various game objects on the game screen.

\subsubsection*{State Variables}
$SCREEN : \text{pygame window object}$\\
\textcolor{red}{w: $\mathbb{Z}$}\\
\textcolor{red}{h: $\mathbb{Z}$}

\subsubsection*{State Invariant}
N/A
\subsubsection*{Assumptions}
N/A
\subsubsection*{Access Routine Semantics}

new \textcolor{red}{\st{SetUpDisplay} WindowSetUp }(\textcolor{red}{width, height})
\begin{itemize}
    \item transition: \\$SCREEN := new\ pygame\ window\ object$\\ \textcolor{red}{w := width}\\ \textcolor{red}{h := height} 
    \item output: $out := self$
    \item exception: None
\end{itemize}

\noindent \textcolor{red}{\st{show()}}:
\begin{itemize}
    \item transition: Display The following contents
    \begin{itemize}
    \item Welcoming message
    \item Display game mode selection
    \item Game instruction
    \item Prevent game addiction message
    \end{itemize}
    \item output: none
    \item exception: None
\end{itemize}

\noindent getScreen()
\begin{itemize}
\item transition: none
\item output: $out := SCREEN$
\item exception: none
\end{itemize}

\noindent setTitle()
\begin{itemize}
\item transition: set Title of the game window
\item output: none
\item exception: none
\end{itemize}

\noindent setIcon()
\begin{itemize}
\item transition: set Icon of the game window
\item output: none
\item exception: none
\end{itemize}

\noindent setBackgroup()
\begin{itemize}
\item transition: set background of the game window
\item output: none
\item exception: none
\end{itemize}

\newpage
%%%%%%%%%%%%%%%%%%%%%%%%%%%%%%%%%%%%%%%%%%%%%%%%%%%%%%%%%%%%%%%%

%%%%%%%%%%%%%%%%%%%%%%%%%%%%SingleController%%%%%%%%%%%%%%%%%%%%
\section{SingleController Module}

\subsection*{Template Module}
SingleController

\subsection*{Uses}
BulletDisplay, MonsterMatrixDisplay, SpaceShipDisplay, ScoreDisplay,
ObstaclesDisplay, \textcolor{red}{pygame.sprite.Spite}

\subsection*{Syntax}
\subsubsection*{Exported Constants}
None
\subsubsection*{Exported Types}
SingleController = ?
\subsubsection*{Exported Access Programs}
\begin{tabular}{| l | l | l | p{5cm} |}
\hline
\textbf{Routine name} & \textbf{In} & \textbf{Out} & \textbf{Exceptions}\\
\hline
new SingleController & pygame window, pygame screen & SingleController & \\
\hline
run & Keyboard Inputs &  & \\
\hline
\textcolor{red}{getScore} &&$\mathbb{Z}$&\\
\hline
\textcolor{red}{doesWin} &&$\mathbb{B}$&\\
\hline
\end{tabular}

\subsection*{Semantics}
\subsubsection*{State Variables}
All the model objects and corresponding display objects.
\subsubsection*{State Invariant}
None
\subsubsection*{Assumptions}
None
\subsubsection*{Access Routine Semantics}
\noindent new SingleController(window, screen):
\begin{itemize}
\item transition: Create the model objects and corresponding display objects.
\item output: $out := \mathit{self}$
\item exception: None
\end{itemize}

\noindent run()
\begin{itemize}
\item transition: The controller should do the following things:
\begin{itemize}
\item Let player move space by pressing $\leftarrow$ and $\rightarrow$.
\item Let player shoot bullet by pressing \verb|SPACE|.
\item If any monster is dead or any game item is shot, let them disappear from the game window. 
\item If any monsters are shot, increase the score.
\item If any game items are shot, do corresponding operations. 
\item If a round is finished, go to the next round.
\item If the spaceship is shot, decrease spaceship lives.
\item If the obstacle is shot, decrease obstacle areas.
\end{itemize}
\item output: none
\item exception: none
\end{itemize}

\noindent \textcolor{red}{getScore()}
\begin{itemize}
\item transition : none
\item output : out := score of this game
\item exception: none
\end{itemize}

\noindent \textcolor{red}{doesWin()}
\begin{itemize}
\item transition : none
\item output : out := if the player wins
\item exception: none
\end{itemize}

\newpage
%%%%%%%%%%%%%%%%%%%%%%%%%%%%%%%%%%%%%%%%%%%%%%%%%%%%%%%%%%%%%%%%

%%%%%%%%%%%%%%%%%%%%%%%DoubleContoller%%%%%%%%%%%%%%%%%%%%%%%%%%
\section{DoubleController Module}

\subsection*{Template Module}
DoubleController

\subsection*{Uses}
BulletDisplay, MonsterMatrixDisplay, SpaceShipDisplay, ScoreDisplay,
ObstaclesDisplay, \textcolor{red}{pygame.sprite.Sprite}

\subsection*{Syntax}
\subsubsection*{Exported Constants}
None
\subsubsection*{Exported Types}
DoubleController = ?
\subsubsection*{Exported Access Programs}
\begin{tabular}{| l | l | l | p{5cm} |}
\hline
\textbf{Routine name} & \textbf{In} & \textbf{Out} & \textbf{Exceptions}\\
\hline
new DoubleController & pygame window, pygame screen  & DoubleController & \\
\hline
run & Keyboard Inputs &  & \\
\hline
\textcolor{red}{getScore} &&$\mathbb{Z}$&\\
\hline
\textcolor{red}{doesWin} &&$\mathbb{B}$&\\
\hline
\end{tabular}

\subsection*{Semantics}
\subsubsection*{State Variables}
All the model objects and corresponding display objects.
\subsubsection*{State Invariant}
None
\subsubsection*{Assumptions}
None
\subsubsection*{Access Routine Semantics}
\noindent new DoubleController(window, screen):
\begin{itemize}
\item transition: Create the model objects and corresponding display objects.
\item output: $out := \mathit{self}$
\item exception: None
\end{itemize}

\noindent run()
\begin{itemize}
\item transition: The controller should do the following things:
\begin{itemize}
\item Let player1 move space by pressing $\leftarrow$ and $\rightarrow$.
\item Let player1 shoot bullet by pressing \verb|SPACE|.
\item Let player2 move space by pressing \verb|A| and \verb|D|.
\item Let player2 shoot bullet by pressing \verb|S|.
\item If any monster is dead or any game item is shot, let them disappear from the game window. 
\item If any monsters are shot, increase the score.
\item If any game items are shot, do corresponding operations. 
\item If a round is finished, go to the next round.
\item If the spaceship is shot, decrease spaceship lives.
\item If the obstacle is shot, decrease obstacle areas.
\end{itemize}
\item output: none
\item exception: none
\end{itemize}



\noindent \textcolor{red}{getScore()}
\begin{itemize}
\item transition : none
\item output : out := score of this game
\item exception: none
\end{itemize}

\noindent \textcolor{red}{doesWin()}
\begin{itemize}
\item transition : none
\item output : out := if the player wins
\item exception: none
\end{itemize}
\newpage
%%%%%%%%%%%%%%%%%%%%%%%%%%%%%%%%%%%%%%%%%%%%%%%%%%%%%%%%%%%%%%%%

%%%%%%%%%%%%%%%%%%%%%%%%%%%%%%%%TotalController%%%%%%%%%%%%%%%%%
\section{TotalController Module}

\subsection*{Template Module}
TotalController

\subsection*{Uses}
\textcolor{red}{\st{SetUpDisplay} WindowSetUp}

\subsection*{Syntax}
\subsubsection*{Exported Constants}
None
\subsubsection*{Exported Types}
TotalController = ?
\subsubsection*{Exported Access Programs}
\begin{tabular}{| l | l | l | p{5cm} |}
\hline
\textbf{Routine name} & \textbf{In} & \textbf{Out} & \textbf{Exceptions}\\
\hline
new TotalController &  & TotalController & \\
\hline
run & Keyboard Input &  & \\
\hline
\end{tabular}

\subsection*{Semantics}
\subsubsection*{State Variables}
$s : \text{WindowSetUp}$
\subsubsection*{State Invariant}
None
\subsubsection*{Assumptions}
None
\subsubsection*{Access Routine Semantics}
\noindent new TotalController():
\begin{itemize}
\item transition: $s := new SetUpDisplay()$
\item output: $out := \mathit{self}$
\item exception: None
\end{itemize}

\noindent run()
\begin{itemize}
\item transition:\\
$s.run()$\\
If user chooses single player model $\Rightarrow$ Invoke SingleController\\
If user chooses double player model $\Rightarrow$ Invoke DoubleController\\
\item output: none
\item exception: none
\end{itemize}
\newpage
%%%%%%%%%%%%%%%%%%%%%%%%%%%%%%%%%%%%%%%%%%%%%%%%%%%%%%%%%%%%%%%%

%%%%%%%%%%%%%%%%%%%%%%%%%%%%%%%%%Diver%%%%%%%%%%%%%%%%%%%%%%%%%%
\section{Driver Module}

\subsection*{Template Module}
Driver

\subsection*{Uses}
TotalController

\subsection*{Syntax}
\subsubsection*{Exported Constants}
None
\subsubsection*{Exported Types}
Driver = ?
\subsubsection*{Exported Access Programs}
\begin{tabular}{| l | l | l | p{5cm} |}
\hline
\textbf{Routine name} & \textbf{In} & \textbf{Out} & \textbf{Exceptions}\\
\hline
new Driver &  & Driver & \\
\hline
run & &  & \\
\hline
\end{tabular}

\subsection*{Semantics}
\subsubsection*{State Variables}
$t : \text{TotalController}$
\subsubsection*{State Invariant}
None
\subsubsection*{Assumptions}
None
\subsubsection*{Access Routine Semantics}
\noindent new Driver():
\begin{itemize}
\item transition: $t := new\ TotalController()$
\item output: $out := \mathit{self}$
\item exception: None
\end{itemize}

\noindent run()
\begin{itemize}
\item transition: $t.run()$
\item output: none
\item exception: none
\end{itemize}
\newpage
%%%%%%%%%%%%%%%%%%%%%%%%%%%%%%%%%%%%%%%%%%%%%%%%%%%%%%%%%%%%%%%%


%%%%%%%%%%%%%%%%%%%%%%%%LifeDisplay%%%%%%%%%%%%%%%%%%%%%%%%%%%%%%%
\section{LifeDisplay Module \textcolor{red}{Newly Added module}}

\subsection*{Template Module}
LifeDisplay

\subsection*{Uses}
\textcolor{red}{pygame.sprite.Sprite}


\subsection*{Syntax}
\subsubsection*{Exported Constants}
None
\subsubsection*{Exported Types}
LifeDisplay = ?
\subsubsection*{Exported Access Programs}
\begin{tabular}{| l | l | l | p{5cm} |}
\hline
\textbf{Routine name} & \textbf{In} & \textbf{Out} & \textbf{Exceptions}\\
\hline
new LifeDisplay & pygame screen & LifeDisplay & \\
\hline
show & $\mathbb{R}, \mathbb{R}, \mathbb{Z}, \mathbb{Z}$ &  & \\
\hline
\end{tabular}

\subsection*{Semantics}
\subsubsection*{State Variables}
$SCREEN : \text{pygame window object}$
\subsubsection*{State Invariant}
None
\subsubsection*{Assumptions}
None
\subsubsection*{Access Routine Semantics}
\noindent new LifeDisplay(screen):
\begin{itemize}
\item transition: $SCREEN := screen$
\item output: $out := \mathit{self}$
\item exception: None
\end{itemize}

\noindent show(x, y, life, index)
\begin{itemize}
\item transition: display life at (x, y) \# Index is used to indicate which 
spaceship's life to be displayed
\item output: none
\item exception: none
\end{itemize}
\newpage
%%%%%%%%%%%%%%%%%%%%%%%%%%%%%%%%%%%%%%%%%%%%%%%%%%%%%%%%%%%%%%%%%%%


%%%%%%%%%%%%%%%%%%%%%%%%%%Welcome message%%%%%%%%%%%%%%%%%%%%%%%%%%%
\section{WelcomeMessageDisplay Module \textcolor{red}{Newly Added module}}

\subsection*{Template Module}
WelcomeMessageDisplay
\subsection*{Uses}
\textcolor{red}{pygame.sprite.Sprite}

\subsection*{Syntax}
\subsubsection*{Exported Constants}
None
\subsubsection*{Exported Types}
WelcomeMessageDisplay = ?
\subsubsection*{Exported Access Programs}
\begin{tabular}{| l | l | l | p{5cm} |}
\hline
\textbf{Routine name} & \textbf{In} & \textbf{Out} & \textbf{Exceptions}\\
\hline
new WelcomeMessageDisplay & pygame screen &WelcomeMessageDisplay & \\
\hline

\end{tabular}

\subsection*{Semantics}
\subsubsection*{State Variables}
None
\subsubsection*{State Invariant}
None
\subsubsection*{Assumptions}
None
\subsubsection*{Access Routine Semantics}
\noindent new WelcomeMessageDisplay(screen):
\begin{itemize}
\item transition: display welcome message at screen
\item output: $out := \mathit{self}$
\item exception: None
\end{itemize}
\newpage
%%%%%%%%%%%%%%%%%%%%%%%%%%%%%%%%%%%%%%%%%%%%%%%%%%%%%%%%%%%%%%%%%%%


%%%%%%%%%%%%%%%%%%%%%%%%%%%%%%%%%ModeSelection%%%%%%%%%%%%%%%%%%%%
\section{ModeSelectionDisplay Module \textcolor{red}{Newly Added module}}

\subsection*{Template Module}
ModeSelectionDisplay

\subsection*{Uses}
\textcolor{red}{pygame.sprite.Sprite}

\subsection*{Syntax}
\subsubsection*{Exported Constants}
None
\subsubsection*{Exported Types}
ModeSelectionDisplay = ?
\subsubsection*{Exported Access Programs}
\begin{tabular}{| l | l | l | p{5cm} |}
\hline
\textbf{Routine name} & \textbf{In} & \textbf{Out} & \textbf{Exceptions}\\
\hline
new ModeSelectionDisplay & pygame screen &ModeSelectionDisplay & \\
\hline

\end{tabular}

\subsection*{Semantics}
\subsubsection*{State Variables}
None
\subsubsection*{State Invariant}
None
\subsubsection*{Assumptions}
None
\subsubsection*{Access Routine Semantics}
\noindent new ModeSelectionDisplay(screen):
\begin{itemize}
\item transition: display mode selection message at screen
\item output: $out := \mathit{self}$
\item exception: None
\end{itemize}
\newpage
%%%%%%%%%%%%%%%%%%%%%%%%%%%%%%%%%%%%%%%%%%%%%%%%%%%%%%%%%%%%%%%%%%

%%%%%%%%%%%%%%%%%%%%%%%%%%%%%%Game Instruction %%%%%%%%%%%%%%%%%%%
\section{GameInstruction Module \textcolor{red}{Newly Added module}}

\subsection*{Template Module}
GameInstruction

\subsection*{Uses}
\textcolor{red}{pygame.sprite.Sprite}

\subsection*{Syntax}
\subsubsection*{Exported Constants}
None
\subsubsection*{Exported Types}
GameInstruction = ?
\subsubsection*{Exported Access Programs}
\begin{tabular}{| l | l | l | p{5cm} |}
\hline
\textbf{Routine name} & \textbf{In} & \textbf{Out} & \textbf{Exceptions}\\
\hline
new GameInstruction & pygame screen, game mode &GameInstruction & \\
\hline

\end{tabular}

\subsection*{Semantics}
\subsubsection*{State Variables}
None
\subsubsection*{State Invariant}
None
\subsubsection*{Assumptions}
None
\subsubsection*{Access Routine Semantics}
\noindent new GameInstruction(screen, mode):
\begin{itemize}
\item transition: display proper game instruction according to mode at screen
\item output: $out := \mathit{self}$
\item exception: None
\end{itemize}
\newpage
%%%%%%%%%%%%%%%%%%%%%%%%%%%%%%%%%%%%%%%%%%%%%%%%%%%%%%%%%%%%%%%%%%

%%%%%%%%%%%%%%%%%%%%%%%%%%%Game item introduction %%%%%%%%%%%%%%%%%%
\section{GameItemIntroductionDisplay Module \textcolor{red}{Newly Added module}}

\subsection*{Template Module}
GameItemIntroductionDisplay

\subsection*{Uses}
\textcolor{red}{pygame.sprite.Sprite}

\subsection*{Syntax}
\subsubsection*{Exported Constants}
None
\subsubsection*{Exported Types}
GameItemIntroductionDisplay = ?
\subsubsection*{Exported Access Programs}
{\scriptsize
\begin{tabular}{| l | l | l | p{5cm} |}
\hline
\textbf{Routine name} & \textbf{In} & \textbf{Out} & \textbf{Exceptions}\\
\hline
new GameItemIntroductionDisplay & pygame screen &GameItemIntroductionDisplay& \\
\hline
\end{tabular}
}
\subsection*{Semantics}
\subsubsection*{State Variables}
None
\subsubsection*{State Invariant}
None
\subsubsection*{Assumptions}
None
\subsubsection*{Access Routine Semantics}
\noindent new GameItemIntroductionDisplay(screen):
\begin{itemize}
\item transition: display game items introduction on screen
\item output: $out := \mathit{self}$
\item exception: None
\end{itemize}
\newpage
%%%%%%%%%%%%%%%%%%%%%%%%%%%%%%%%%%%%%%%%%%%%%%%%%%%%%%%%%%%%%%%%%%



%%%%%Lose Disyplay
\section{LoseDisplay Module \textcolor{red}{Newly Added module}}

\subsection*{Template Module}
LoseDisplay

\subsection*{Uses}
\textcolor{red}{pygame.sprite.Sprite}

\subsection*{Syntax}
\subsubsection*{Exported Constants}
None
\subsubsection*{Exported Types}
LoseDisplay = ?
\subsubsection*{Exported Access Programs}

\begin{tabular}{| l | l | l | p{5cm} |}
\hline
\textbf{Routine name} & \textbf{In} & \textbf{Out} & \textbf{Exceptions}\\
\hline
new LoseDisplay & pygame screen, $\mathbb{Z}$ &LoseDisplay& \\
\hline
\end{tabular}

\subsection*{Semantics}
\subsubsection*{State Variables}
None
\subsubsection*{State Invariant}
None
\subsubsection*{Assumptions}
None
\subsubsection*{Access Routine Semantics}
\noindent new LoseDisplay(screen, score):
\begin{itemize}
\item transition: display lose message and score on screen
\item output: $out := \mathit{self}$
\item exception: None
\end{itemize}
\newpage
%%%%%%%%%%%%%%%%%%

%%%%%Win Disyplay
\section{WinDisplay Module \textcolor{red}{Newly Added module}}

\subsection*{Template Module}
WinDisplay

\subsection*{Uses}
\textcolor{red}{pygame.sprite.Sprite}

\subsection*{Syntax}
\subsubsection*{Exported Constants}
None
\subsubsection*{Exported Types}
WinDisplay = ?
\subsubsection*{Exported Access Programs}

\begin{tabular}{| l | l | l | p{5cm} |}
\hline
\textbf{Routine name} & \textbf{In} & \textbf{Out} & \textbf{Exceptions}\\
\hline
new WinDisplay & pygame screen, $\mathbb{Z}$ &WinDisplay& \\
\hline
\end{tabular}

\subsection*{Semantics}
\subsubsection*{State Variables}
None
\subsubsection*{State Invariant}
None
\subsubsection*{Assumptions}
None
\subsubsection*{Access Routine Semantics}
\noindent new WinDisplay(screen, score):
\begin{itemize}
\item transition: display win message and score on screen
\item output: $out := \mathit{self}$
\item exception: None
\end{itemize}
%%%%%%%%%%%%%%%%%%
\end {document}
