
\documentclass[12pt]{article}


\usepackage{tabularx}
\usepackage{booktabs}
\usepackage{graphicx}
\usepackage{paralist}
\usepackage{listings}
\usepackage{booktabs}
\usepackage{hyperref}
\usepackage{amsfonts}
\usepackage{amsmath}
\usepackage{color}
\usepackage{fancyhdr}
\usepackage{geometry}
\usepackage{multirow}
\geometry{margin = 0.75in}

\title{SE 3XA3: Development Plan}

\begin{document}

\maketitle

%%%%%%%%%%%%%%%%%%%Team Information%%%%%%%%%%%%%%%%%%%%%%
{\Large Team Information:}
\begin{table}[htp]
\centering
{\Large
\begin{tabular}{|c|c|c|}
\hline
\multicolumn{1}{|l|}{Team Number} & Name         & MACID   \\ \hline
\multirow{3}{*}{L03 G07}          & Qianlin Chen & chenq84 \\ \cline{2-3} 
                                  & Jiacheng Wu  & wuj187  \\ \cline{2-3} 
                                  & Tingyu Shi   & shit19  \\ \hline
\end{tabular}
}
\end{table}
%%%%%%%%%%%%%%%%%%%%%%%%%%%%%%%%%%%%%%%%%%%%%%%%%%%%%%

\begin{table}[htp]
\caption{Revision History} 
\begin{tabularx}{\textwidth}{llX}
\toprule
\textbf{Date} & \textbf{Developer(s)} & \textbf{Change}\\
\midrule
January 31, 2022 & All team members & Create Outline\\
February 3, 2022 & All team members & Write contents\\
\bottomrule
\end{tabularx}
\end{table}

\newpage

\section{Introduction}
\noindent This document is about the
development plan of space invaders project
based at the original project at the
following link:\\
\url{https://github.com/leerob/space-invaders}
\section{Team Information}
{\Large Team Information:}
\begin{table}[htp]
\centering
{\Large
\begin{tabular}{|c|c|c|}
\hline
\multicolumn{1}{|l|}{Team Number} & Name         & MACID   \\ \hline
\multirow{3}{*}{L03 G07}          & Qianlin Chen & chenq84 \\ \cline{2-3} 
                                  & Jiacheng Wu  & wuj187  \\ \cline{2-3} 
                                  & Tingyu Shi   & shit19  \\ \hline
\end{tabular}
}
\end{table}
\section{Team Meeting Plan}
\subsection{Time, Location, Frequency and Contents of Meetings}
Meetings will take place every Monday and Wednesday from 9:30 am to 11:20 am. All the meetings will be on the Microsoft teams
before February 7. We created the Microsoft Team
channel "3XA3 Lab03" and we will work at the group 7 in this channel. After
February 7, our team 
will hold
in person meetings.
If regular meetings are
not enough, we will hold extra meetings
according to team members' available time.

\noindent The contents of the meetings
include discussion, implementation,
assigning works outside the meetings and
reporting the working progress. All the
decisions and modifications should be
recorded and updated by the end of the day
of meetings. 
\subsection{Member Roles for Meetings}
The following are member 
roles for meetings:
\begin{itemize}
\item Tingyu Shi is the chair of the
meetings. He will host all the
meetings and gather participants' ideas.
\item Jiacheng Wu will be the recorder of the
meeting. He will record all the decisions
and changes to the project or documents.
\item Qianlin Chen will assign all the tasks
after the meetings to all the team members.
\end{itemize}
\subsection{Rules for Agendas}
\noindent An agenda is supposed to be made before every meeting. The following
are rules for agendas of meetings:
\begin{itemize}
\item All team members are supposed to attend the meetings. If any member cannot attend the meeting, he/she needs to tell the other teammates at least one day before the meeting.
\item Topics must be determined before every meeting.
\item The length of meeting time shall be determined before the meeting.
\item There is only one chair/leader in the meeting.
\item Conflicts in the meeting shall be recorded and solved before the next meeting.
\item All members should communicate respectfully during the meeting.
\item All members shall be assigned a “take home” work after the meeting. The estimated time of the work of each member shall be close.
\item First topic will be reviewing the agenda.
\item Assess all team members’ contributions in every meetings.
\end{itemize}
\section{Team Communication Plan}
Our team will use Microsoft Team and
Facebook for communication. Microsoft Team
will be used for big tasks like
meetings, implementation and distributing
tasks. Facebook will be used for small problems like
debugging, advising and small modifications.
Also, the team will contact with TAs and
professor via Microsoft Team or Email. GitLab is also
used to do the project. All members should write detailed commit messages so that
others can understand the modifications.


\section{Team Member Roles}
Each member in the team will have multiple roles since the team is small. Tingyu Shi will be the leader of our team. Tingyu Shi will decide all the decisions and modifications for this project. Also, he will gather all the modifications using git. Qianlin Chen will be the lead designer of the project. Jiacheng Wu is in charge of UI of the game. All of the member will be involved in Development, testing, Documentation. 

\begin{table}[h]
    \centering
    \begin{tabular}{|c|c|}
    \hline
         Member Names & Roles/Expert  \\
         \hline
         Tingyu Shi & \shortstack{ Team Leader, Project Manager, Meeting host \\ Developer, Documentation }\\
         \hline
         Qianlin Chen & \shortstack{ Lead designer, Tester, Developer, \\ Documentation}\\
         \hline
         Jiacheng Wu & \shortstack{UI designer, Tester, Developer, \\ Documentation} \\
         \hline
    \end{tabular}
    %\caption{Caption}
    \label{tab:label1}
\end{table}

\section{Git Workflow Plan}
The git workflow we decided to use is Feature-Branch Workflow. The following are
some details of how it will work:
\begin{itemize}
\item The whole project has a mean branch. 
You can view this branch as the official 
branch for this project. Main branch should
not contain broken code. The code can only
be merged to main branch after testing.
\item For each new feature, a new branch 
should be created. For example, for our space invaders project, we can have one 
branch for space ship to move and one
branch for space ship to catch additional
game items.
\item After each feature is tested, team 
members can merge them to the main branch.
\end{itemize}
The main advantage for this workflow is that 
main branch will not contain any broken code.\\
Labels will be used to indicate issues.\\
Milestones will be used to track progress,
for example, team members can track if 
indicated issues are resolved.
\section{Proof of Concept Demonstration Plan}
\subsection{Some Significant Risks}
\subsubsection{Implementation Difficulties}
This game will be implemented with python.
The good news is that every team member has
experience with this language. However, we need to use pygame module for implementation and only one team member has experience with this module. Therefore, other two team members should spend time to learn this module and familiarize with the API.
\subsubsection{Testing}
Game testing is usually harder compared with
testing of other projects. The main difficulty here is to choose an appropriate
testing method so that it can cover all the scenarios.
\subsubsection{Library Installation}
Library installation is not a problem in this project.
All users can use pip to install pygame library easily.
\subsubsection{Portability}
We believe that source code can be executed
in different operating systems. However, the
problem is to make sure that executable files
can be executed in different operating systems.
\subsubsection{Design and Requirements Difficulties}
\begin{itemize}
\item Design Difficulty: The main difficulty in design process is designing appropriate modules. We need to make sure 
that each module is low coupling and high cohesion.
\item Requirements Difficulty: The main difficulty for requirements engineering is 
to solve conflicts and make sure the final
requirements document does not have any conflicts. 
\end{itemize}
\subsection{Plans to Overcome Risks}
\subsubsection{Implementation}
Make sure team members who are not familiar 
with pygame go through YouTube tutorials
before implementation process.
\subsubsection{Testing}
Unit testing will be critical for this project, before merging new features to the 
main branch, make sure they are well tested.\\
Before doing the final test for the whole
project, team members should first document
possible scenarios and make sure all the 
scenarios can be tested.\\
We can also invite some friends to play 
this game without showing them source code.
This is the same as black-box testing. 
\subsubsection{Portability}
Team members will generate different executable
files for different operating systems and run these files in different operating systems.
\subsubsection{Design and Requirements}
\begin{itemize}
\item Some of the design problems can only
be found when we are coding. Therefore, 
we need to review and revise design document regularly while implementing.
\item After finishing the requirement documents, we
will hold a meeting to do inspection to review them. 
\end{itemize}
\section{Technology}
The following are some technology details 
for this project:
\begin{itemize}
\item Programming language: Python
\item IDE: Each team member can use different IDEs according to their preferences. VSCode is recommended.
\item Testing Framework: Pytest will be 
used for testing.
\item Documentation: All documents should
be written using \LaTeX. Doxygen will be 
used to document code. 
\end{itemize}
\section{Coding style}
The project’s coding structure follows Python3, so we decide to use the linting standards set by VSCode in order to detect bugs and style problems in Python source code. Because of the dynamic nature of Python, some warnings may be incorrect. However, spurious warnings should be fairly infrequent. 

\noindent Moreover, we will obey the all Python Style Rules listed in the \url{https://google.github.io/styleguide/pyguide.html} (simply like the limitation of line length and comments’ rules).

\section{Project schedule}
Gantt\_Chart\_L03\_G07.gan file can be found at the following:\\ 
\href{https://gitlab.cas.mcmaster.ca/shit19/2022_winter_3xa3_l03_g07/-/blob/main/ProjectSchedule/Gantt_Chart_L03_G07.gan}{\textcolor{red}{.gan file Link}}\\

\noindent Gantt\_Chart\_L03\_G07.pdf file can be found at the following:\\ \href{https://gitlab.cas.mcmaster.ca/shit19/2022_winter_3xa3_l03_g07/-/blob/main/ProjectSchedule/Gantt_Chart_L03_G07.pdf}{\textcolor{red}{.pdf file Link}}\\

\noindent The above two files can be found at 
the folder called "Gantt\_Chart\_Files", which
is located at the DevelopmentPlan folder.\\
For some future documents, subtasks are not 
specified since we do not know the template 
right now. Gantt Chart will be updated in the 
future.  
\section{Project review}
This part will be available after revision 1.
\end {document}